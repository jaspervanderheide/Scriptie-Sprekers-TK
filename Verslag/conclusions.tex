\section{Conclusies}
\label{sec:conc}

Dit onderzoek vindt een \textit{accuracy} van XXX en een $F_1$ score van XXX voor het classificeren van spreekbeurten in de Tweede Kamer naar partij-affiliatie. De baseline scores zijn respectievelijk 0.11 en 0.15. Als rekening wordt gehouden met namen van partijen en Kamerleden, daalt de \textit{accuracy} naar XXX en de $F_1$ score. Dit onderzoek vindt aanwijzingen dat deze classificatie afhankelijk is van de partij-status (oppositie of regering). Als rekening wordt gehouden met woordgebruik van individuele Kamerleden, daalt de nauwkeurigheid verder naar.... Hoewel dit onderzoek hoge scores vindt voor classificatie, lijken deze in grote mate afhankelijk te zijn van andere factoren dan ideologie.