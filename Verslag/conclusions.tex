\section{Conclusies}
\label{sec:conc}

Dit onderzoek vindt een \textit{accuracy} en $F_1$ score van 0.80 voor het classificeren van spreekbeurten in de Tweede Kamer naar partij-affiliatie. De beste classificatiemethode maakt gebruik van Support-Vector Machines De baseline scores zijn respectievelijk 0.11 en 0.15. Als rekening wordt gehouden met partijnamen en achternamen Kamerleden daalt de \textit{accuracy} naar 0.58 en de $F_1$ score naar 0.57. Dit onderzoek vindt aanwijzingen dat deze classificatie afhankelijk is van de partij-status (oppositie of regering). Daarnaast vindt dit onderzoek geen aanwijzingen dat de classificatie afhankelijk is van positie op de links-rechts as. Als rekening wordt gehouden met woordgebruik van individuele Kamerleden, dalen de \textit{accuracy} en $F_1$ verder naar 0.27. Daarmee lijkt de classificatie naar partij-affiliatie in grote mate niet het gevolg van ideologie. Deze conclusie trekt daarmee de bruikbaarheid van tekstclassificatie voor het vinden van een relatie tussen woordgebruik en ideologie in twijfel. Op een aantal punten wijken de bevindingen van dit onderzoek af van vergelijkbare onderzoeken \cite{Hirst_textto, diermeier_godbout_yu_kaufmann_2012}. Voor een vervolgonderzoek kan dit onderzoek uitgebreid worden met een aantal aanbevelingen.