\documentclass[a4paper,pdf]{article} % gebruik acm style voor je scriptie: 
%\documentclass[format=manuscript, screen=true, review=false, pdf]{acmart} 
\usepackage{amsmath}
\usepackage{amsfonts}
\usepackage{amssymb}
\usepackage{hyperref}
\usepackage{diagbox}
\usepackage{pdfpages} % http://mirror.unl.edu/ctan/macros/latex/contrib/pdfpages/pdfpages.pdf
\usepackage{booktabs} 
\setlength\parindent{24pt}
\usepackage{makecell}
\usepackage{siunitx}
\usepackage[T1]{fontenc}
\usepackage[utf8]{inputenc}
\usepackage{amsmath}
\usepackage{multirow}
\usepackage{graphicx}
\usepackage{subfig}
\usepackage[numbers]{natbib}
\usepackage[colorinlistoftodos]{todonotes} % handig voor commentaar: gebruik \todo{}, zie ftp://ftp.fu-berlin.de/tex/CTAN/macros/latex/contrib/todonotes/todonotes.pdf
\usepackage{listings}
\usepackage{pdfpages}
\usepackage{tcolorbox}
\usepackage{float}
\usepackage{caption}
%\usepackage{subcaption}

\usepackage[bottom]{footmisc}
% when writing in Dutch
\usepackage[dutch]{babel}
\selectlanguage{dutch}


% linenumbering  See https://texblog.org/2012/02/08/adding-line-numbers-to-documents/
%\usepackage{lineno}
%\linenumbers


\begin{document}
%\input{titlepage}
%\documentclass[]{article}
%\usepackage{lmodern}
%%\usepackage{fontspec}
%\usepackage{amssymb,amsmath}
%\usepackage{ifxetex,ifluatex}
%\usepackage{fixltx2e} % provides \textsubscript
%\ifnum 0\ifxetex 1\fi\ifluatex 1\fi=0 % if pdftex
%  \usepackage[T1]{fontenc}
%  \usepackage[utf8]{inputenc}
%\else % if luatex or xelatex
%  \ifxetex
%    \usepackage{mathspec}
%    \usepackage{xltxtra,xunicode}
%  \else
%    \usepackage{fontspec}
%  \fi
%  \defaultfontfeatures{Mapping=tex-text,Scale=MatchLowercase}
%  \newcommand{\euro}{€}
%\fi
%% use upquote if available, for straight quotes in verbatim environments
%\IfFileExists{upquote.sty}{\usepackage{upquote}}{}
%% use microtype if available
%\IfFileExists{microtype.sty}{%
%\usepackage{microtype}
%\UseMicrotypeSet[protrusion]{basicmath} % disable protrusion for tt fonts
%}{}
%\usepackage{graphicx}
%\makeatletter
%\def\maxwidth{\ifdim\Gin@nat@width>\linewidth\linewidth\else\Gin@nat@width\fi}
%\def\maxheight{\ifdim\Gin@nat@height>\textheight\textheight\else\Gin@nat@height\fi}
%\makeatother
%% Scale images if necessary, so that they will not overflow the page
%% margins by default, and it is still possible to overwrite the defaults
%% using explicit options in \includegraphics[width, height, ...]{}
%\setkeys{Gin}{width=\maxwidth,height=\maxheight,keepaspectratio}
%\ifxetex
%  \usepackage[setpagesize=false, % page size defined by xetex
%              unicode=false, % unicode breaks when used with xetex
%              xetex]{hyperref}
%\else
%  \usepackage[unicode=true]{hyperref}
%\fi
%\hypersetup{breaklinks=true,
%            bookmarks=true,
%            pdfauthor={},
%            pdftitle={},
%            colorlinks=true,
%            citecolor=blue,
%            urlcolor=blue,
%            linkcolor=magenta,
%            pdfborder={0 0 0}}
%\urlstyle{same}  % don't use monospace font for urls
%\setlength{\parindent}{0pt}
%\setlength{\parskip}{6pt plus 2pt minus 1pt}
%\setlength{\emergencystretch}{3em}  % prevent overfull lines
%\setcounter{secnumdepth}{0}
%
%\date{}
%
%\begin{document}


\begin{titlepage}


\begin{center}
 
\textsc{\Large   Ideologie en classificatie in de Handelingen van de Tweede Kamer}

\bigskip

\textsc{\large
submitted in partial fulfillment for the degree of bachelor of science\\
%
\bigskip
Jasper van der Heide\\
%
10732721\\
%
\bigskip
Bachelor Informatiekunde\\
%
Faculteit der Natuurwetenschappen, Wiskunde en Informatica\\
%
Universiteit van Amsterdam\\
%
\bigskip
2018-06-28
}

\end{center}
 

\vfill

% In case of an internal project, remove External Supervisor or if you had two internal supervisors, change the header into 
%  & First Supervisor & Second Supervisor  \\
\begin{center}
\begin{tabular}{|l||ll|}
\hline
 & \textbf{Begeleider} & \textbf{Tweede lezer}  \\   
 \hline
\textbf{Titel, Naam} & Dr Maarten Marx&  \\
\textbf{Affliatie} &UvA, FNWI, IvI & \\ 
\textbf{Email} & maartenmarx@uva.nl& . \\
\hline
\end{tabular}
\end{center}


\bigskip

% logos
\begin{center}
\mbox{\includegraphics[width=.2\paperwidth]{logo-uva.png} 
%\includegraphics[width=.2\paperwidth]{TitlePages/logos/ads.png}

}
\end{center}
\end{titlepage}

%
%\newpage
%
%\end{document}
  % or use another template
\begin{abstract}
In verschillende onderzoeken zijn parlementaire teksten geclassificeerd naar partij-affiliatie. Dit onderzoek heeft eerst gezocht naar de beste classificatiemethode voor het Nederlandse parlement. Vervolgens is gekeken in hoeverre de classificatie het gevolg is van ideologie. Hiervoor is gekeken naar de invloed van namen, in regering of oppositie zitten, positie op de links-rechts as en woordgebruik van sprekers.\par
De beste classificatiemethode met een nauwkeurigheid van 0.80 is Support Vector Machine. Dit daalt naar 0.58 als achternamen van Kamerleden en partijnamen weggehaald worden. Het onderzoek vond ook aanwijzingen dat de classificatie afhankelijk is van of een partij in regering of oppositie zit. Aanwijzingen voor afhankelijkheid van positie op links-rechts as zijn daarentegen niet gevonden. Als laatste daalt de nauwkeurigheid verder naar 0.27 als Kamerleden verdeeld worden over de training en test set, wat suggereert dat de oorspronkelijke classificatie afhankelijk was van woordgebruik van sprekers. Dit leidt tot de conclusie dat in grote mate de classificatie niet het gevolg is van ideologie.
\end{abstract}
\pagebreak
\tableofcontents
\pagebreak




% Here you input all your sections in seperate files

\section{Introductie}
\label{sec:intro}
Teksten van politieke partijen kunnen dienen als bron voor het bepalen van ideologische positie van andere teksten, aangezien zij zowel tekst bevatten als ook een bekende ideologie in de vorm van een partij van de spreker; de partij-affiliatie. Het classificeren op basis van tekst kan inzichten geven over de relatie tussen ideologie en woordgebruik. Deze informatie kan vervolgens toegepast worden op andere teksten die wellicht ideologisch van aard zijn. Bijvoorbeeld kan men aan de hand van deze informatie teksten uit kranten classificeren op basis van ideologie \cite{DBLP:journals/corr/Biessmann16,Hirst_textto}.\par
In diverse landen zijn al onderzoeken gedaan naar het classificeren naar partij-affiliatie op basis van teksten van politici \cite{DBLP:journals/corr/Biessmann16,Ferreira2016UsingTT}. Met deze tekstclassificatie naar partij-affiliatie proberen onderzoekers uit te vinden in hoeverre ideologie terug te vinden is in teksten van politici. De resultaten van de tekstclassificaties zijn in alle gevallen ruim boven de baseline. Diverse onderzoeken wijzen daarentegen ook naar redenen dat dit niet alleen het gevolg is van ideologie. Zo suggereren de  resultaten van Hirst et al. \cite{Hirst_textto} dat de partij-status (oppositie tegenover regering) van invloed is op de classificatie. Daarnaast laat dit onderzoek ook zien dat partijnamen een grote invloed hebben op de classificatie.\par
Een onderzoek gericht op het Nederlandse parlement is niet gevonden. Ook beperken veel onderzoeken zich vaak tot één classificatiemethode.\par
Dit onderzoek richt zich daarom op meerdere classificatiemethoden en daarnaast specifiek op de Nederlandse politiek. De onderzoeksvraag luidt daarom dus ook: "In hoeverre is classificatie op basis van partij-affiliatie aan de hand van spreekbeurten in de Tweede Kamer het gevolg van ideologie?"\par
Deze vraag wordt beantwoord door de antwoorden te vinden op de volgende deelvragen:
\begin{enumerate}
    \item Wat is het beste classificatiemodel voor classificatie naar partij-affiliatie in de Tweede Kamer en wat is het resultaat van dit model?
    \item In hoeverre is deze classificatie afhankelijk van achternamen van Kamerleden en partijen?
    \item In hoeverre wordt deze classificatie bepaald door partij-status (oppositie of regering)?
    \item In hoeverre wordt deze classificatie bepaald door links/rechts positie?
    \item In hoeverre wordt deze classificatie bepaald door woordgebruik van sprekers?
\end{enumerate}
Voor de eerste deelvraag zullen Support Vector Machine, Logistische Regressie en Naive Bayes met verschillende parameters vergeleken worden aan de hand van \textit{accuracy} en $F_1$ score. Bij de tweede deelvraag wordt gekeken naar classificatie zonder achternamen van Kamerleden en partijnamen of met alleen achternamen van Kamerleden en partijnamen. De derde vraag bestaat uit meerdere experimenten, waarin gekeken zal worden naar de hoeveelheid misclassificaties binnen regering of oppositie tegenover tussen regering en oppositie. Daarnaast zal gekeken worden naar overlap in woordgebruik binnen regering en verschil in scores als een partij gewisseld is van partij-status.


\paragraph{Overzicht van scriptie}
Sectie 2 bevat vergelijkbare onderzoeken in andere parlementen. Sectie 3 bevat de methodologie van de verschillende deelvragen. Sectie 4 bevat de resultaten. Sectie 5 bevat de evaluatie van zowel de resultaten als de methodologie. Sectie 6 bevat ten slotte het antwoord op de onderzoeksvraag.

\section{Gerelateerd werk}
\label{sec:rel}

Diermeier et al. deden onderzoek naar het classificeren op basis van ideologische positie in de Amerikaanse Senaat\cite{diermeier_godbout_yu_kaufmann_2012}. Ze trainden hun classicatie op de speeches van de 25 meest liberale en de 25 meest conservatieve senatoren van het 101e tot en met het 107e congres en testten op de 25 meest liberale en de 25 meest conservatieve senatoren van het 108e congres. Later in het onderzoek vergeleken ze ook de 25 gematigd conservatieve  en de 25 gematigd liberale senatoren. Voor classificatie maakten ze gebruik van support vector machines. Verder maakten ze gebruik van TF-IDF met een minimale woordfrequentie van 50 en een documentfrequentie van 10, \textit{Part-Of-Speech tagging} en werden alle eigennamen verwijderd. Dit onderzoek wist de ideologie van de senatoren te voorspellen met een 94 procent nauwkeurigheid voor de classificatie van de extremen, maar slechts een 52 procent nauwkeurigheid voor de classificatie van de gematigde senatoren.\par
Als een vervolg op dit onderzoek deden Graeme Hirst et al. een vergelijkbaar onderzoek naar zowel het Canadese als het Europese Parlement\cite{Hirst_textto}. In dit onderzoek maken zij gebruik van support-vector machines. In tegenstelling tot het onderzoek van Diermeier et al., vinden zij minder dat de woorden van de sprekers een uiting zijn van ideologie. Daarentegen vinden zij wel een grotere invloed van oppositie tegenover regering in de woorden van de sprekers.\par
Ferreira probeerde interventies van politici te classificeren op basis van geslacht, leeftijdsgroep, partij-affiliatie en ori\"{e}ntatie\cite{Ferreira2016UsingTT} In dit onderzoek maakt hij gebruik van twee classificatiemethoden, Logistische regressie en MIRA. Logistische regressie werd aangevuld met \textit{group Lasso} regularisatie. Voor wegingen van woorden werd gebruikt gemaakt van woordfrequentie, TF-IDF, $\Delta$-TF-IDF, $\Delta$-BM-25. Daarnaast wordt er gebruik gemaakt van woordclustering, \textit{Concise Semantic Analysis} en stylometrische eigenschappen. Op \textit{Part-Of-Speech tagging} na hadden stylometrische eigenschappen een duidelijke negatieve invloed op de classificatie. In alle classificaties kon men aan de hand van logistische regressie en \textit{group Lasso} regularisatie een F1-score van 0.87 of hoger bereiken.\par


\subsection{RQ1}

\subsection{RQ2}

\section{Methodology}
\label{sec:meth}


\subsection{Description of the data}
De data die gebruikt wordt, zijn de Handelingen van de Tweede Kamer gedurende het niet-demissionaire kabinet Rutte 2 (5 november 2012 tot 22 maart 2017). Deze data is in xml-formaat van de website officielebekendmakingen.nl gehaald, samen met corresponderende metadata xml-bestanden. De bestanden van de Handelingen bevatten voornamelijk informatie over spreekbeurten tijdens een debat, waaronder naam van een spreker, partij-affiliatie en inhoud van de spreekbeurt. Deze gegevens zijn samengevoegd tot een tabel en opgeslagen als csv-bestand.\par
Deze dataset bevat naast de verkozen partijen van de 2012 tweede kamerverkiezingen, ook afsplitsingen van die partijen (tien in totaal) en bezoeken van vertegenwoordigingen van die partijen uit de Eerste Kamer (10 in totaal). Omdat van beide categori\"{e}n relatief weinig data is en er overlap zit met hun oorspronkelijke partij, zijn deze er uit gehaald.

\begin{table}[H]
\centering
\caption{Spreekbeurten per partij}
\label{Sprpartij}
\begin{tabular}{|l|l|}
\hline
Partij       & Aantal spreekbeurten \\ \hline
SP           & 27034 \\ \hline
D66          & 24600 \\ \hline
VVD          & 22990 \\ \hline
CDA          & 22452 \\ \hline
PvdA         & 22217 \\ \hline
PVV          & 16408 \\ \hline
GroenLinks   & 12954 \\ \hline
ChristenUnie & 11401 \\ \hline
SGP          & 6316  \\ \hline
PvdD         & 4081  \\ \hline
50PLUS       & 2223  \\ \hline
\end{tabular}
\end{table}

\pagebreak
\subsection{Wat plotjes en tabelletjes}

Zie het IPython Notebook \url{PandasAndLatex.ipynb} voor de code om vanuit pandas een poltje op te slaan en een dataframe als tabel op te slaan. Het werkt ideaal! 

De interrupties van Wilders staan beschreven in Figure~\ref{fig:wilders} en Tabel~~\ref{tab:Wilders}.




\pagebreak



\pagebreak
\subsection{Methods}
Hoe je je vraag gaat beantwoorden.


Dit is de langste sectie van je scriptie. 

Als iets erg technisch wordt kan je een deel naar de Appendix verplaatsen. 

Probeer er een lopend verhaal van te maken.

Het is heel handig dit ook weer op te delen nav je deelvragen:

\subsubsection{RQ1}

\subsubsection{RQ2}


\section{Resultaten}

\subsection{DV1: Beste classificatiemethode}
In figuur \ref{fig:scores} zijn de uitslagen van de grid search te zien. Hierin is te zien dat SVM en logistische regressie beide hoge nauwkeurigheden behalen, maar dat logistische regressie ook veel lage nauwkeurigheden haalt tussen 0 en 0.1. Naive Bayes zit tussen de 0.25 en 0.6.
\begin{figure}[H]
  \centering
    \includegraphics[width=0.50\paperwidth]{Verslag/Tables/scores.png}
\caption{Histogram van de grid search met de $F_1$ scores van de classificatiemethoden}
\label{fig:scores}
\end{figure}
Het beste resultaat werd bereikt met Support Vector Machine gebruikmakend van \textit{stochastic gradient descent learning} en L2 regularisatie. In de grid search behaalde deze methode een $F_1$ score en nauwkeurigheid van 0.75. Voor beide scores was dit het hoogste van de grid search. De woorden waren hierbij gestemd. De features waren zowel unigrams, bigrams als trigrams. Geen features zijn weggelaten door minimale of maximale documentfrequenties. De waarden van deze features waren \textit{tf-idf} scores. Het maximum aantal iteraties was 5 voor de grid search, maar de rest van resultaten zijn op basis van 100 iteraties.\par
Tabel \ref{tab:classrapport} laat de scores zien per partij met het aantal documenten in de test set. De nauwkeurigheid voor deze classificatie is 0.80. De $F_1$ scores per partij liggen tussen de 0.7 en 0.9. De partijen met een sterke focus op één onderwerp, 50PLUS, PVV en PvdD, als ook de SGP hebben hoge scores. De coalitiepartijen, VVD en PvdA, daarentegen hebben lagere scores. Figuur \ref{fig:confusionmatrix} laat zien waar de fouten in deze classificatie zitten. De meest karakteristieke n-grams per partij zijn te zien in tabel \ref{tab:MostImportantWords}. Met meest karakteristiek worden de n-grams bedoeld die de hoogste coëfficiënt hebben in de classificatie en die dus relatief het meeste belangrijk zijn voor de classificatie van een partij. Hierin is te zien dat vrijwel alle n-grams achternamen van Kamerleden of partijnamen bevatten.\par

\begin{table}[H]
\caption{Classificatie scores per partij van de beste classificatiemethode (SVM). Gemiddelde van vijfmaal kruisvalidatie.}
\label{tab:classrapport}
\centering
\begin{tabular}{lrrrr}
\toprule
{} &  Precision &  Recall &  F1\_score &  Documenten \\
Partij       &            &         &           &             \\
\midrule
50PLUS       &      0.920 &   0.852 &     0.884 &        77.0 \\
   CDA       &      0.760 &   0.774 &     0.768 &       378.8 \\
ChristenUnie &      0.806 &   0.780 &     0.788 &       210.4 \\
   D66       &      0.806 &   0.704 &     0.746 &       401.6 \\
  GroenLinks &      0.824 &   0.752 &     0.784 &       215.8 \\
   PVV       &      0.752 &   0.850 &     0.800 &       344.4 \\
  PvdA       &      0.696 &   0.658 &     0.670 &       351.0 \\
  PvdD       &      0.868 &   0.842 &     0.852 &        86.8 \\
   SGP       &      0.782 &   0.802 &     0.788 &       133.0 \\
    SP       &      0.756 &   0.816 &     0.784 &       457.8 \\
   VVD       &      0.718 &   0.696 &     0.704 &       323.4 \\
 avg / total &      0.768 &   0.762 &     0.762 &      2980.0 \\
\bottomrule
\end{tabular}

\end{table}


\begin{figure}[H]
  \centering
    \includegraphics[width=0.50\paperwidth]{Verslag/Tables/confusionmatrix.png}
\caption{Confusion matrix van de beste classificatiemethode (SVM). Gemiddelde van vijfmaal kruisvalidatie.}
\label{fig:confusionmatrix}
\end{figure}



\begin{table}[H]
\caption{Meest karakteristieke n-grams per partij op basis van de beste classificatiemethode (SVM) gedurende kabinet-Rutte II. N-grams die niet achternamen van Kamerleden of partijnamen bevatten, zijn dikgedrukt. }
\label{tab:MostImportantWords} 
\centering
\hspace*{-1in}
\begin{tabular}{lllll}
\toprule
          50PLUS &               CDA &         ChristenUnie &                  D66 &         GroenLinks \\
\midrule
          50plus &               cda &      de christenunie &                  d66 &         groenlinks \\
    het lid krol &           het cda &         christenunie &         mijn fractie &   lid van tongeren \\
        lid krol &            de cda &        lid dik faber &  leden van veldhoven &   het lid voortman \\
   lid krol naar &       cda fractie &              lid dik &        van veldhoven &  lid voortman naar \\
   krol naar mij &    de cda fractie &          het lid dik &            veldhoven &       lid voortman \\
       krol naar &   het lid omtzigt &            dik faber &    lid van veldhoven &       van tongeren \\
            krol &       lid omtzigt &     leden voordewind &       lid van meenen &           tongeren \\
      van 50plus &  lid omtzigt naar &  de leden voordewind &               d66 is &           voortman \\
 gepensioneerden &  omtzigt naar mij &                faber &    de leden bergkamp &  de leden voortman \\
         ouderen &      omtzigt naar &                  dik &       leden bergkamp &     leden voortman \\
\bottomrule
\end{tabular}
 
\end{table} 
\addtocounter{table}{-1} 
\begin{table}[H]
\caption{Meest karakteristieke n-grams per partij op basis van de beste classificatiemethode (SVM) gedurende kabinet-Rutte II. N-grams die niet achternamen van Kamerleden of partijnamen bevatten, zijn dikgedrukt.  \emph{(Vervolg)}} 
\centering
\hspace*{-1.3in}
\begin{tabular}{llllll}
\toprule
              PVV &             PvdA &              PvdD &              SGP &             SP &             VVD \\
\midrule
              pvv &          de pvda &      lid ouwehand &              sgp &             sp &          de vvd \\
           de pvv &             pvda &  lid ouwehand nar &           de sgp &          de sp &             vvd \\
      islamitisch &    de partij van &  het lid ouwehand &      sgp fractie &   lid van gerv &  de vvd fractie \\
        lid graus &    van de arbeid &      ouwehand nar &   de sgp fractie &     sp fractie &     vvd fractie \\
    het lid graus &        de arbeid &  ouwehand nar mij &     led dijkgraf &  de sp fractie &       de vvd is \\
    lid graus nar &    partij van de &          ouwehand &  de led dijkgraf &   van gerv nar &          vvd is \\
          miljard &       partij van &       vor de dier &      led van der &       gerv nar &      vor de vvd \\
    graus nar mij &     pvda fractie &           de dier &  de led bisschop &   gerv nar mij &      wat de vvd \\
        graus nar &           arbeid &              dier &     led bisschop &           gerv &     vvd betreft \\
 madlener nar mij &  de pvda fractie &     de partij vor &           sgp is &       van gerv &  de vvd betreft \\
\bottomrule
\end{tabular}
 
\end{table}

\subsection{DV2: Invloed van namen}
In tabel \ref{tab:MostImportantWords} was al te zien dat de meest karakteristieke n-grams voornamelijk achternamen van Kamerleden of partijnamen bevatten. In tabel \ref{tab:rapportonlynames} zijn de scores te zien voor een classificatie met alleen achternamen van Kamerleden en partijnamen. De nauwkeurigheid is 0.61. De scores zijn gedaald ten opzichte van de resultaten van deelvraag 1, maar hoger dan de baseline scores.\par
\begin{table}[H]
\caption{Classificatierapport van beste classificatie met alleen achternamen van Kamerleden en partijnamen. Hiervoor is alleen gebruikgemaakt van unigrams. Gemiddelde van vijfmaal kruisvalidatie.}
\label{tab:rapportonlynames}
\centering
\begin{tabular}{lrrr}
\toprule
{} &  Precisie &  Sensitiviteit &  $F_1$ score \\
\midrule
50PLUS       &       0.82 &    0.88 &      0.85\\
PvdD         &       0.68 &    0.78 &      0.69 \\
GroenLinks   &       0.71 &    0.66 &      0.68  \\
PVV          &       0.66 &    0.71 &      0.67  \\
CDA          &       0.67 &    0.65 &      0.66  \\
ChristenUnie &       0.66 &    0.58 &      0.62 \\
SP           &       0.61 &    0.64 &      0.62  \\
VVD          &       0.68 &    0.57 &      0.62  \\
SGP          &       0.69 &    0.54 &      0.60  \\
D66          &       0.56 &    0.53 &      0.54  \\
PvdA         &       0.56 &    0.51 &      0.52  \\
\midrule
Totaal       &       0.64 &    0.62 &      0.62  \\
\bottomrule
\end{tabular}

\end{table}

In tabel \ref{tab:rapportwithoutnames} zijn de $F_1$ scores te zien van classificatie met achternamen van Kamerleden en partijnamen vervangen. De nauwkeurigheid hiervan is 0.58. De scores zijn lager dan die uit deelvraag 1 en lager dan van de classificatie met alleen namen. Wel zijn de scores nog steeds hoger dan de baseline. In tabel \ref{tab:MostImportantWordsWithoutNames} is vervolgens te zien welke n-grams het meest karakteristiek zijn per partij voor deze classificatie.\par
\begin{table}[H]
\caption{Classificatie scores per partij van beste classificatiemethode (SVM) uit deelvraag 1 zonder achternamen van Kamerleden en partijnamen met het relatieve verschil in $F_1$ score ten opzichte van tabel \ref{tab:classrapport}. Gemiddelde van vijfmaal kruisvalidatie.}
\label{tab:rapportwithoutnames}
\centering
\begin{tabular}{lcccc}
\toprule
{} &  Precision &  Recall &  $F_1$ score &  $\Delta F_1$ score (\%) \\
\midrule
SGP          &       0.71 &    0.73 &      0.72 & -18 \\
PvdD         &       0.75 &    0.70 &      0.72 & -19 \\
PVV          &       0.63 &    0.80 &      0.70 & -19 \\
ChristenUnie &       0.68 &    0.46 &      0.55 & -21 \\
CDA          &       0.52 &    0.53 &      0.52 & -23 \\
SP           &       0.54 &    0.71 &      0.61 & -24 \\
D66          &       0.55 &    0.55 &      0.55 & -28 \\
VVD          &       0.54 &    0.49 &      0.52 & -30 \\
50PLUS       &       0.86 &    0.49 &      0.62 & -32 \\
PvdA         &       0.51 &    0.48 &      0.50 & -32 \\
GroenLinks   &       0.64 &    0.38 &      0.48 & -41 \\
\midrule
Totaal       &       0.59 &    0.58 &      0.57 &        -29 \\
\bottomrule
\end{tabular}

\end{table}

\begin{table}[H] 
\caption{Meest karakteristieke n-grams per partij op basis van de classificatiemethode (SVM) uit deelvraag 1 zonder achternamen van Kamerleden en partijnamen gedurende kabinet-Rutte II.} 
\label{tab:MostImportantWordsWithoutNames} 
\centering
\hspace*{-0.8in}
\begin{tabular}{lllll}
\toprule
                 50PLUS &             CDA &        ChristenUnie &           D66 &              GroenLinks \\
\midrule
                ouderen &  PARTIJ fractie &       vluchtelingen &  mijn fractie &                     zou \\
        gepensioneerden &        inwoners &           inderdaad &          mijn &       kamer hierover te \\
               plussers &          PARTIJ &        mensenhandel &    natuurlijk &        persoonsgebonden \\
 koopkrachtontwikkeling &        regering &              zullen &       fractie &            in elk geval \\
                 oudere &     de regering &            gezinnen &   het kabinet &               elk geval \\
                     50 &            echt &  voedselverspilling &       vandaag &                  in elk \\
            50 plussers &         fractie &          constateer &  buitengewoon &             hierover te \\
              werkenden &            hier &       ik constateer &    belangrijk &     belastingontwijking \\
        overwegende dat &             wij &          onder meer &      minister &             regering om \\
            overwegende &            zeer &         begeleiding &       kabinet &  hierover te informeren \\
\bottomrule
\end{tabular}
 
\end{table} 
\addtocounter{table}{-1} 
\begin{table}[H] 
\caption{Meest karakteristieke n-grams per partij op basis van de classificatiemethode (SVM) uit deelvraag 1 zonder achternamen van Kamerleden en partijnamen gedurende kabinet-Rutte II. \emph{(Vervolg)}} 
\centering
\hspace*{-1.3in}
\begin{tabular}{llllll}
\toprule
              PVV &         PvdA &              PvdD &                    SGP &            SP &            VVD \\
\midrule
      islamitisch &       kinder &              dier &  mevrouw de voorzitter &          zegt &     volgen mij \\
        nederland &       jonger &          de natur &             mevrouw de &       huurder &  yyyyy fractie \\
          miljard &  mijn partij &            de bio &               dank zer &          mens &         volgen \\
          brussel &      gezamen &     bio industrie &                   punt &     de bevolk &          yyyyy \\
 belastingbetaler &        beter &     constater dat &           eenverdiener &        bevolk &         veilig \\
               al &       tevred &               bio &                    wel &        bezuin &       regelgev \\
            islam &      collega &  de bio industrie &            bewindslied &    bestuurder &     essentieel \\
         de islam &          die &            milieu &               allerlei &  bureaucratie &       aangegev \\
          allemal &          sam &         industrie &                     je &          toch &     ondernemer \\
          mijnher &    circulair &         constater &      vor de beantwoord &    mening dat &          kader \\
\bottomrule
\end{tabular}
 
\end{table}



\subsection{DV3: Oppositie of regering}
In figuur \ref{fig:distributies} zijn de distributies van de errors, zoals gedefinieerd in formule \ref{eq:error} te zien van combinaties van regerings- en oppositiepartijen. Bijgevoegd zijn het aantal combinaties (N), het gemiddelde ($\mu$) en de standaarddeviatie ($\sigma$).
\begin{figure}[H]
    \centering
    \hspace*{-0.2in}
    \subfloat[Tussen twee regeringspartijen (N=200, \protect\\Mediaan=25.83, IKA=9.18)]{{\includegraphics[width=7cm]{Verslag/Handmatig/Regering.png} }}%
    \subfloat[Tussen twee oppositiepartijen (N=8100, \protect\\Mediaan=-1.09, IKA=4.98)]{{\includegraphics[width=7cm]{Verslag/Handmatig/Oppositie.png} }}\quad
    \hspace*{-0.2in}
    \subfloat[Tussen een regeringspartij en een oppositiepartij (N=4000, Mediaan=-2.92, IKA=6.82)]{{\includegraphics[width=7cm]{Verslag/Handmatig/Mix.png} }}%
    \subfloat[Totaal (N=12100, Mediaan=-1.40, IKA=5.76)]{{\includegraphics[width=7cm]{Verslag/Handmatig/Totaal.png} }}\quad
    \caption{Genormaliseerde distributie van de error uit formule \ref{eq:error} voor de verschillende combinaties.}%
    \label{fig:distributies}%
\end{figure}
Voor alle distributies was de nulhypothese verworpen dat deze normaal verdeeld zijn ($p < 0.01$) door middel van een normaalheidstoets. In tabel \ref{tab:whitney} is te zien dat er een significant verschil is tussen de distributies binnen regering en binnen oppositie tegenover de distributie tussen een regeringspartij en een oppositiepartij. Binnen regeringspartijen zijn er gemiddeld 26.11 misclassificaties meer dan verwacht en binnen oppositiepartijen gemiddeld 0.43.

\begin{table}[H]
\caption{Uitslagen van eenzijdige Mann-whitneytoets tussen de distributie tussen een regeringspartij en oppositiepartij en twee distributies. $\alpha$ is 0.01.}
\label{tab:whitney}
\centering
\begin{tabular}{lrr}
\toprule
{} &  $p$-waarde &  $U$-waarde\\
\midrule
Tussen twee regeringspartijen       &       \num{7.04e-124} &    717042 \\
Tussen twee oppositiepartijen         &       \num{4.4e-108} &    16328471 \\
\bottomrule
\end{tabular}
\end{table}
In tabel \ref{tab:WoordenBalkenende4} zijn de meest karakteristieke n-grams te zien voor classificatie van kabinet-Balkenende IV. Hierin zijn geen opvallende overlappen te zien van regeringspartijen met de classificatie van kabinet-Rutte II in tabel \ref{tab:MostImportantWordsWithoutNames}.
\begin{table}[H] 
\caption{Meest karakteristieke n-grams per partij op basis van beste classificatiemethode uit deelvraag 1 zonder achternamen van Kamerleden en partijnamen gedurende kabinet-Balkenende IV.} 
\label{tab:WoordenBalkenende4} 
\centering
\hspace*{-0.6in}
\begin{tabular}{lllll}
\toprule
              CDA &        ChristenUnie &              D66 &          GroenLinks &         PVV \\
\midrule
   PARTIJ fractie &  fractie van PARTIJ &          premier &       PARTIJfractie &     burgers \\
              wij &      de fractie van &       de premier &  fractie van PARTIJ &        land \\
          fractie &          de fractie &              hij &             premier &  de burgers \\
       wij hebben &         fractie van &          ik hoop &          de fractie &      burger \\
             dank &       verschillende &            jaren &      de fractie van &        onze \\
         KAMERLID &        mijn fractie &             hoop &         fractie van &        deze \\
          overleg &       beantwoording &           europa &           politieke &      gewoon \\
       aangegeven &                blij &   schone energie &                  ik &  immigratie \\
     buitengewoon &              moment &  de arbeidsmarkt &                 mij &        niet \\
 algemeen overleg &           discussie &          plannen &                deal &  natuurlijk \\
\bottomrule
\end{tabular}
 
\end{table} 
\addtocounter{table}{-1} 
\begin{table}[H] 
\caption{Meest karakteristieke n-grams per partij op basis van beste classificatiemethode uit deelvraag 1 zonder achternamen van Kamerleden en partijnamen gedurende kabinet-Balkenende IV.\emph{(Vervolg)}} 
\centering
\hspace*{-0.4in}
\begin{tabular}{lllll}
\toprule
       PvdA &              PvdD &                SGP &            SP &                        VVD \\
\midrule
        wij &            dieren &       mijn fractie &        mensen &                     PARTIJ \\
    vrouwen &     bio industrie &                wel &       leraren &             PARTIJ fractie \\
 belangrijk &  de bio industrie &      beantwoording &          niet &               onze fractie \\
         of &            de bio &  voorzitter ik wil &          zegt &                    fractie \\
  onderzoek &               bio &      de voorzitter &     personeel &                         je \\
      weten &       veehouderij &               toch &    leerlingen &                      markt \\
     vragen &            natuur &   de bewindslieden &            is &  voorzitter PARTIJ fractie \\
        roc &     dierenwelzijn &      bewindslieden &    informatie &                     in elk \\
       goed &    de veehouderij &    dankzeggen voor &     onderwijs &                ondernemers \\
   kinderen &         industrie &         dankzeggen &  bureaucratie &               in elk geval \\
\bottomrule
\end{tabular}
 
\end{table}
In tabel \ref{RegeringOppositie} zijn de scores te zien van de classificatie die getraind is op een zittingsperiode, maar getest op een andere. De resultaten zijn gedaald, maar nog boven de baseline. De daling verschilt per partij en zittingsperiode met dalingen van $F_1$ scores tussen 12 en 92\%. \par
\begin{table}[H]
\caption{$F_1$ scores van de classificatie getraind op de dataset van Balkenende IV of Rutte II (minus 50PLUS) en getest op de ander. Scores van een classificatie getraind en getest op kabinet-Rutte II zonder 50PLUS zijn bijgevoegd ter referentie, als ook de relatieve daling. De classificatiemethode uit deelvraag 1 is gebruikt zonder achternamen van Kamerleden en partijnamen. Partijen met een asterisk zijn gewisseld van partij-status.}
\centering
\hspace*{-0.2in}
\label{RegeringOppositie}
\begin{tabular}{llll}
\toprule
{}& {}&\multicolumn{2}{l}{Training set$\rightarrow$ Test set}\\
\midrule
{} &Rutte II &\makecell{Balkenende IV $\rightarrow$ Rutte II\\Baseline = 0.11}&  \makecell{Rutte II $\rightarrow$ Balkenende IV\\Baseline = 0.12}  \\
\midrule
CDA*          &0.53&       0.28 &    0.43  \\
ChristenUnie* &0.55&       0.37 &    0.22  \\
D66          &0.54&       0.16 &    0.28  \\
GroenLinks   &0.49&       0.31 &    0.04  \\
PVV          &0.70&       0.50 &    0.60  \\
PvdA         &0.52&       0.29 &    0.27  \\
PvdD         &0.73&       0.64 &    0.45 \\
SGP          &0.74&       0.56 &    0.49  \\
SP           &0.61&       0.41 &    0.53 \\
VVD*          &0.51&       0.18 &    0.10  \\ \hline
Totaal       &0.58&       0.34 &    0.35 \\
\bottomrule
\end{tabular}

\end{table}


\subsection{DV4: Links-rechts as}
In tabel \ref{fig:distanceerror} is de error te zien ten opzichte van de afstand op de links-rechts as.
\begin{figure}[H]
  \centering
    \includegraphics[width=0.60\paperwidth]{Verslag/Tables/Ideology.png}
\caption{Error ten opzichte van de afstand op de links-rechts as van twee partijen. Gebaseerd op 100 classificaties met verschillende test en train set. De Pearson correlatie is 0.09 en de $p$-waarde \num{2.39e-20}.}
\label{fig:distanceerror}
\end{figure}
De Pearson correlatie van 0.09 is daarmee met een $p$-waarde van \num{2.39e-20} significant op het significantieniveau van 0.01, maar wel positief gecorreleerd. Uit deelvraag 3 bleek dat de error binnen oppositie of regering significant afweek van de error tussen regering en oppositie. Dit effect lijkt ook zichtbaar in figuur \ref{fig:distanceerror}. Daarom is er ook gekeken naar de correlatie tussen afstand op de links-rechts as en error binnen oppositie en tussen regerings- en oppositiepartij. De resultaten zijn te zien in tabel \ref{tab:pearson}. Beide correlaties zijn statistische significant op het significantieniveau van 0.01, maar in tegengestelde richting.

\begin{table}[H]
\caption{Pearson correlatie tussen error en afstand op de links-rechts as voor combinaties van partij-status.}
\label{tab:pearson}
\centering
\begin{tabular}{lrr}
\toprule
{} &  Pearson correlatie &  $p$-waarde\\
\midrule
Tussen oppositie- en regeringspartij       &       -0.29 &    \num{3.44e-69} \\
Tussen twee oppositiepartijen         &       0.18 &    \num{1.76e-55} \\
\bottomrule
\end{tabular}
\end{table}

\subsection{DV5: Woordgebruik van sprekers}
In tabel \ref{tab:rapporttaalgebruik} staan de scores van classificatie waarbij de Kamerleden verdeeld zijn over de training en test set. De scores zijn hierbij nauwelijks hoger dan de baseline.
\begin{table}[H]
\caption{Classificatierapport van beste classificatiemethode uit deelvraag 1 zonder achternamen van Kamerleden en partijnamen met de Kamerleden verdeeld over training en test set. Gemiddelde van tienmaal kruisvalidatie.}
\label{tab:rapporttaalgebruik}
\centering
\begin{tabular}{lrrrr}
\toprule
{} &  Precision &  Recall &  $F_1$ score &  $\Delta F_1$ score (\%) \\
\midrule
50PLUS       &       0.29 &    0.06 &      0.09 &           \\
CDA          &       0.12 &    0.20 &      0.14 &          \\
ChristenUnie &       0.08 &    0.14 &      0.09 &           \\
D66          &       0.22 &    0.22 &      0.22 &          \\
GroenLinks   &       0.16 &    0.04 &      0.05 &          \\
PVV          &       0.29 &    0.50 &      0.37 &          \\
PvdA         &       0.25 &    0.19 &      0.21 &          \\
PvdD         &       0.46 &    0.17 &      0.22 &          \\
SGP          &       0.17 &    0.05 &      0.07 &           \\
SP           &       0.34 &    0.33 &      0.33 &          \\
VVD          &       0.31 &    0.26 &      0.24 &          \\
\midrule
Totaal       &       0.31 &    0.24 &      0.24 &         \\
\bottomrule
\end{tabular}

\end{table}


\section{Discussie}
\subsection{DV1: Beste classificatiemethode}
Het onderzoek behaalt resultaten in lijn der verwachting op basis van gerelateerd werk en daarnaast ruim boven de baseline scores. De lage scores voor de coalitiepartijen steunen de hypothese van een afhankelijkheid van partij-status, zoals besproken wordt in deelvraag 3. Het bijna alleen voorkomen van namen van partijen en Kamerleden in de meest karakteristieke n-grams per partij in tabel \ref{tab:MostImportantWords} steunt daarnaast het vermoeden dat deze classificatie sterk afhankelijk is van die namen, zoals besproken wordt in deelvraag 2.\par
Dit onderzoek heeft zich beperkt tot methoden genoemd in vergelijkbare onderzoeken en waarvan de implementatie beschikbaar is in scikit-learn. Een aantal methoden die in gerelateerde literatuur leidden tot goede classificaties zijn daarom niet getest. Ook nieuwe methoden die nog niet gebruikt zijn in een vergelijkbaar onderzoek voor politieke tekst classificatie zijn daarom niet getest. Daarnaast richtte zich dit ook maar op een beperkt aantal parameterwaarden. Voor vervolgonderzoek kan daarom dit onderdeel uitgebreid worden. Het effect van het beperkte maximum iteraties was bij de beste classificatiemethode 2\%.\par
Het onderzoek van Hirst et al. vond dat resultaten afhankelijk kunnen zijn van documentgrootte. Alle documenten in dit onderzoek zijn kleiner dan de grootste documentgrootte uit het onderzoek van Hirst et al. en ook de minimale documentgrootte ligt lager dan de kleinste documentgrootte uit dat onderzoek.
Het effect wat zij vinden tussen documentgrootte van 267 en 6666 is een verschil in \textit{accuracy} van 19.8\%. Dit onderzoek vindt inderdaad dat kleinere documenten vaker foutief geclassificeerd worden.
\begin{figure}[H]
  \centering
    \includegraphics[width=0.40\paperwidth]{Verslag/Tables/misclassifiedlengths.png}
\caption{Genormaliseerde distributie van documentlengtes van foutief geclassificeerde documenten en alle documenten. Totaal van 5-fold cross-validation, waardoor documenten vaker voor kunnen komen. Mediaan documentlengte van foutief geclassificeerde documenten is 321 en voor alle documenten 386.}
\label{fig:misclassified}
\end{figure}
Voor een vervolgonderzoek kan uitgebreider gekeken worden naar dit effect en wat dit betekent voor de resultaten. Het percentage documenten van een vragenuur is tweemaal zo hoog bij foutief geclassificeerde documenten, maar dit lijk te komen doordat deze documenten vaak kleiner zijn (mediaan is 286).\par
Er is verder nog gekeken naar andere verbanden tussen documenten die verkeerd zijn geclassificeerd. Daarbij is nog te zien dat sprekers met weinig documenten relatief iets meer voorkomen in verkeerd geclassificeerde documenten.
\begin{figure}[H]
  \centering
    \includegraphics[width=0.50\paperwidth]{Verslag/Tables/misclassifiedsprekers.png}
\caption{Aantal misclassificaties gedeeld door totaal aantal documenten per spreker tegenover totaal aantal documenten van een spreker. Misclassificaties zijn totaal van 5-fold cross-validation, waardoor documenten vaker mee kunnen tellen. De pearson correlatie is -0.28 en de p-waarde \num{1.07e-4}.}
\label{fig:misclassifiedsprekers}
\end{figure}
Dit versterkt het vermoeden dat de classificatie mede plaatsvindt op basis van woordgebruik van individuele sprekers, zoals besproken wordt in deelvraag 5.\par


\subsection{DV2: Invloed van namen}
De resultaten laten zien dat de classificatie sterk afhankelijk is van partijnamen en achternamen van Kamerleden. De hogere scores voor de classificatie met alleen namen dan zonder namen in combinatie met de woorden in tabel \ref{tab:MostImportantWords} suggereert dat dit het belangrijkste was in de classificatie van deelvraag 1. Deze daling was te verwachten op basis van gerelateerd werk.\par
De n-grams in tabel \ref{tab:MostImportantWordsWithoutNames} komen bij veel partijen overeen met hun ideologie, vooral bij de partijen met een sterke focus op één onderwerp; PVV, PvdD en 50PLUS. Daarnaast zijn er ook n-grams die niet veel over ideologie lijken te zeggen, zoals; \textit{volgens mij}, \textit{ik constateer} en \textit{in elk geval}. Vooral de SGP heeft n-grams die niet veel lijken te zeggen over de ideologie, hoewel deze partij desalniettemin de hoogste $F_1$ score heeft. Met name opvallend hierbij is \textit{mevrouw de voorzitter}, aangezien deze woorden door alle partijen gebruikt worden om via de voorzitter te praten. Voor een vervolgonderzoek kan gekeken naar waarom deze n-grams zo karakteristiek zijn voor partijen.\par
De classificatiemethode die gebruikt is in deze deelvraag, is gebaseerd op de beste methode voor de dataset uit deelvraag 1. Hierin was gevonden dat een combinatie van uni-, bi- en trigrams het beste resultaat opleverde. In tabel \ref{tab:MostImportantWords} is te zien dat trigrams behoren tot de meest karakteristieke n-grams, hoewel de woorden in trigrams vaak overlappen met uni- en bigrams. In tabel \ref{tab:MostImportantWordsWithoutNames} daarentegen zijn er nog maar een paar trigrams, welke grotendeels procedurele zinnen zijn of toevoeging van een lidwoord op een uni- of bigram. Dit verschil suggereert dat trigrams minder belangrijk zijn in de classificatie zonder de namen, dus de classificatiemethode uit deelvraag 1 niet het beste is voor deze classificatie. In vervolgonderzoek kan de opzet van deelvraag 1 toegepast worden op de classificatie zonder de namen, om zo te komen tot een classificatiemethode die het beste resultaat oplevert op de classificatie zonder namen.\par
Er is ook gekeken naar andere namen in de lijst van 100 meest karakteristieke woorden per partij, zoals van gebieden, bedrijven of bewindspersonen. Bewindspersonen komen hier niet in voor. Er komen een aantal gebieden in voor, zoals \textit{aruba}, \textit{limburg} en \textit{saoedi arabië}. Ook komen er organisaties als \textit{gvo} \textit{hvo} en \textit{monsanto} in voor. Deze woorden lijken in sommige gevallen een weerspiegeling te zijn voor ideologie, dus voor vervolgonderzoek lijkt het niet nodig te zijn deze te verwijderen.\par

\subsection{DV3: Oppositie of regering}
In tabel \ref{tab:classrapport} is het opvallend dat de coalitiepartijen lage scores krijgen. Daarnaast laat figuur \ref{fig:confusionmatrix} zien dat er een hoge overlap zit tussen deze twee partijen.\par
De statistische toetsresultaten in tabel \ref{tab:whitney} laten zien dat inderdaad de error groter is binnen oppositie of regering dan tussen een regerings- en oppositiepartij. Dit suggereert dat inderdaad partij-status invloed heeft op de classificatie. \par
De verwachting was dat de error normaal verdeeld zou zijn. De verdelingen uit figuur \ref{fig:distributies} hebben globaal wel de vorm van een normaal verdeling. In figuur \ref{fig:correlation} is het daarnaast opvallend dat partijen zoals SP en PVV ruim onder de regressielijn zitten, terwijl andere partijen er een stuk boven zitten. Dit geeft aanleiding te vermoeden dat er naast het aantal documenten van een partij nog meer factoren van invloed zijn op het aantal misclassificaties en daarmee de verwachte waarde. En deze verwachte waarde en de daar uit volgende error zijn een belangrijke aanname van deze methode. Voor deze methode is het dus belangrijk uit te vinden of dit een goede benadering is van de verwachte waarde. In deelvraag 4 wordt gekeken of links/rechts positie hier nog invloed heeft. Voor een vervolgonderzoek kan nog verder gekeken worden naar invloeden op verwachte waarde of andere confounding biases.\par
De overlap van 100 meest karakteristieke n-grams tussen regeringspartijen die niet voorkomen bij oppositiepartijen gedurende kabinet-Rutte II beperkt zich tot de woorden \textit{en} en \textit{blij}, als ook \textit{toezegging} voor VVD en \textit{toezeggingen} voor PvdA.\par
\begin{table}[H]
\label{tab:overlapkabinetten}
\caption{N-grams die bij minimaal één regeringspartij in beide kabinetten voorkomen in de 100 meest karakteristieke n-grams, maar niet voor één van de twee partijen tijdens het andere kabinet.}
\centering
\begin{tabular}{|l|l|l|l|}
\toprule
&&\multicolumn{2}{c|}{Kabinet-Rutte II} \\ \hline
      &&   PvdA &    VVD\\ \hline

\parbox[t]{2mm}{\multirow{4}{*}{\rotatebox[origin=c]{90}{Kabinet-Balkenende IV}}}&         CDA &    \makecell[l]{\textit{toezeggingen}\\\textit{hun}\\\textit{collega KAMERLID}\\\textit{in}\\\textit{aanpak}\\\textit{collega}} &            \makecell[l]{\textit{algemeen}\\\textit{algemeen overleg}\\\textit{toezegging}\\\textit{helder}\\\textit{overleg}\\\textit{aangegeven}\\\textit{voor}\\\textit{voor PARTIJ}} \\ \cline{2-4} 
 &ChristenUnie &  \makecell[l]{\textit{mijn}\\\textit{waarop}\\\textit{blij}\\\textit{collega KAMERLID}\\\textit{erg}} &        \makecell[l]{\textit{gaan}\\\textit{termijn}\\\textit{blij met de}\\\textit{volgens}\\\textit{volgens mij}\\\textit{blij}\\\textit{beantwoording}}  \\ \cline{2-4} 
  &PvdA &   & \makecell[l]{\textit{volgens}\\\textit{volgens mij}}           \\
\bottomrule
\end{tabular} 
\end{table}
Hoewel er een aantal overeenkomsten zijn qua meest karakteristieke n-grams tussen regeringspartijen van de twee kabinetten, lijkt dit beperkt. De meeste overeenkomsten lijken daarnaast niet heel inhoudelijk gerelateerd aan partij-status. Deze resultaten suggereren daarom ook maar een beperkte invloed van partij-status op de classificatie. Voor een vervolgonderzoek kan uitgebreider gekeken worden naar de overlappende meest karakteristieke n-grams en wat deze zeggen over een regeringspartij.\par
De scores in tabel \ref{RegeringOppositie} laten een duidelijke daling zien ten opzichte van een classificatie van alleen kabinet-Rutte II. Deze algemene daling kan verklaard worden door verschuiving in ideologie, verschil in woordgebruik, verandering van onderwerpen en/of verandering in aantal documenten per partij. De daling is het grootst bij VVD, maar valt mee bij de twee andere partijen die gewisseld zijn van partij-status, ChristenUnie en CDA. Daarnaast is de daling ook heel sterk bij oppositiepartijen GroenLinks en D66, alsook de regeringspartij in beide kabinetten, PvdA. Dat de daling niet consequent groter is bij partijen die gewisseld zijn van partij-status, suggereert dat de invloed van partij-status beperkt is op de classificatie.\par
Dat de experimenten uit Hirst et al. in hun onderzoek wel invloed vinden, maar in dit onderzoek niet kan komen doordat hun onderzoek zich richt op binaire classificatie, terwijl dit onderzoek meerdere partijen heeft. Zo kan het ontbreken van gemeenschappelijke n-grams komen doordat regeringspartijen zich ook van elkaar moeten onderscheiden in dit onderzoek, waarvoor n-grams die relevant zijn voor partij-status weinig effect hebben, terwijl in het onderzoek van Hirst et al. de regeringspartij alleen onderscheiden hoeft te worden van de oppositiepartij. Daarnaast verklaarden zij dat de daling tussen twee zittingsperioden het gevolg was van de wisseling van partij-status. In dit onderzoek kon daarentegen gekeken worden naar effecten op partijen niet die niet van partij-status zijn gewisseld. Hierin was te zien dat de daling ook aanwezig was bij partijen die niet gewisseld zijn van partij-status.\par

\subsection{DV4: Links-rechts as}
De correlatie was tegen de verwachting in positief, waardoor de nulhypothese niet verworpen kan worden. Een deel van deze positieve correlatie lijkt te wijten aan de error tussen de twee regeringspartijen. Daarnaast is het opvallend dat tussen oppositiepartijen de correlatie ook positief is, maar tussen oppositie en regeringspartij juist, zoals eigenlijk verwacht, negatief. Een verklaring hiervoor is niet gevonden. \par
Alle correlaties zijn statistisch significant, maar de Pearson correlatie en daarmee effectgrootte is klein. Daarnaast is het ook opvallend dat de twee combinaties van partij-statussen een andere correlatierichting hebben. Dit suggereert dat de statistische significantie het gevolg is van de grote steekproef en maar een klein effect \cite{Hair}.\par
Er zijn verschillende visies op links en rechts en de indeling van partijen op die as. Daarnaast zijn er nog meerdere assen waarlangs partijen vergeleken kunnen worden. Bijvoorbeeld op basis van conservatief en progressief. Een vervolgonderzoek kan uitgebreider kijken naar welke assen relevant zijn voor partijen in de Tweede Kamer en in hoeverre deze invloed hebben op de classificatie. \par

\subsection{DV5: Woordgebruik van sprekers}
De resultaten uit tabel \ref{tab:rapporttaalgebruik} zijn laag, amper hoger dan de baseline. Dit suggereert inderdaad dat eerdere classificaties in grote mate toch afhankelijk waren van het woordgebruik van sprekers. Dit is opmerkelijk aangezien vergelijkbare onderzoeken dit effect niet vinden. De meest karakteristieke n-grams van deze classificatie wijken daarnaast grotendeels niet af van die uit tabel \ref{tab:MostImportantWordsWithoutNames}.\par
Een alternatieve verklaring is dat de classificatie nu mede op basis van woordvoerderschap is. Per onderwerp heeft een partij vaak maar één woordvoerder, met uitzonderingen van wijzigingen in de fractie. Het is aannemelijk dat het taalgebruik afhankelijk is van woordvoerderschap, aangezien er andere termen gebruikt worden bij bijvoorbeeld een debat over zorg dan bij een debat over onderwijs. Als een woordvoerder op een bepaald onderwerp van een partij in de test set voorkomt, is er een grote kans dat geen enkele spreker van die partij eerder over dat onderwerp heeft gepraat, want de woordvoerder gaat nou eenmaal daarover. Daardoor heeft deze spreker veel n-grams die ook voorkomen bij andere woordvoerders over dat onderwerp, maar van andere partij. Als deze n-grams ook belangrijk zijn voor de classificatie kan het zijn dat de woordvoerder geclassificeerd wordt bij een partij van een andere woordvoerder. Een vervolgonderzoek kan kijken of dit een verklaring is.\par
Vergelijkbare onderzoeken vermijden dit mogelijke probleem door alle spreekbeurten van een spreker samen te voegen tot één document. Zoals al eerder vermeld is dit onpraktisch voor de kleinere partijen. Voor een vervolgonderzoek kan desalniettemin gekeken worden naar deze methode om te kijken of dat wel een weerspiegeling is van ideologische verschillen.\par

\subsection{Algemeen}
Het vergelijken van deze resultaten met vergelijkbaar werk is ingewikkeld, aangezien de keuzes en eigenschappen van die onderzoeken het niet een één-op-één vergelijking maken. Voorbeelden hiervan zijn de taal, het parlement, de documentgrootte, baselines, behouden of weglaten van namen, een spreker als document zien en het trainen en testen op dezelfde spreker. Hoewel de resultaten in sommige gevallen lager zijn dan die uit vergelijkbaar werk, is het belangrijk hier rekening mee te houden. Een vervolgonderzoek zou daarom dit onderzoek kunnen reproduceren op een ander parlement om daarmee te kunnen vergelijken.\par
Dit onderzoek richtte zich hoofdzakelijk op de Handelingen gedurende kabinet-Rutte II. Om te kijken in hoeverre het mogelijk is om deze conclusie door te trekken naar de algemene Handelingen van de Tweede Kamer, kan er in vervolgonderzoek gekeken worden naar meerdere zittingsperioden. Ook kan gekeken worden naar veranderingen als een kabinet demissionair is.\par
Dit onderzoek heeft een aantal beperkingen die in dit hoofdstuk besproken zijn. Het uitvoeren van deze aanbevelingen kan de validiteit en betrouwbaarheid van dit onderzoek vergroten. Ook is dit onderzoek moeilijk te vergelijken met andere onderzoeken om diverse redenen, maar vooral ook omdat het toegepast is op een ander parlement. Desalniettemin geeft dit onderzoek reden om te twijfelen aan de bruikbaarheid van tekstclassificatie van de Handelingen van de Tweede Kamer voor een relatie tussen woordgebruik en ideologie. Daarnaast levert dit onderzoek ook kritieken op een aantal vergelijkbare onderzoeken.\par
\section{Conclusies}
\label{sec:conc}


% your refs
\bibliographystyle{apa}%Used BibTeX style is unsrt
\bibliography{bibliography}


\end{document}
