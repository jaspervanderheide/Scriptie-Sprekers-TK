\section{Methodology}
\label{sec:meth}


\subsection{Description of the data}
De data die gebruikt wordt, zijn de Handelingen van de Tweede Kamer gedurende het niet-demissionaire kabinet Rutte 2 (5 november 2012 tot 22 maart 2017). Deze data is in xml-formaat van de website officielebekendmakingen.nl gehaald, samen met corresponderende metadata xml-bestanden. De bestanden van de Handelingen bevatten voornamelijk informatie over spreekbeurten tijdens een debat, waaronder naam van een spreker, partij-affiliatie en inhoud van de spreekbeurt. Deze gegevens zijn samengevoegd tot een tabel en opgeslagen als csv-bestand.\par
Deze dataset bevat naast de verkozen partijen van de 2012 tweede kamerverkiezingen, ook afsplitsingen van die partijen (tien in totaal) en bezoeken van vertegenwoordigingen van die partijen uit de Eerste Kamer (10 in totaal). Omdat van beide categori\"{e}n relatief weinig data is en er overlap zit met hun oorspronkelijke partij, zijn deze er uit gehaald.

\begin{table}[H]
\centering
\caption{Spreekbeurten per partij}
\label{Sprpartij}
\begin{tabular}{|l|l|}
\hline
Partij       & Aantal spreekbeurten \\ \hline
SP           & 27034 \\ \hline
D66          & 24600 \\ \hline
VVD          & 22990 \\ \hline
CDA          & 22452 \\ \hline
PvdA         & 22217 \\ \hline
PVV          & 16408 \\ \hline
GroenLinks   & 12954 \\ \hline
ChristenUnie & 11401 \\ \hline
SGP          & 6316  \\ \hline
PvdD         & 4081  \\ \hline
50PLUS       & 2223  \\ \hline
\end{tabular}
\end{table}

\pagebreak
\subsection{Wat plotjes en tabelletjes}

Zie het IPython Notebook \url{PandasAndLatex.ipynb} voor de code om vanuit pandas een poltje op te slaan en een dataframe als tabel op te slaan. Het werkt ideaal! 

De interrupties van Wilders staan beschreven in Figure~\ref{fig:wilders} en Tabel~~\ref{tab:Wilders}.




\pagebreak



\pagebreak
\subsection{Methods}
Hoe je je vraag gaat beantwoorden.


Dit is de langste sectie van je scriptie. 

Als iets erg technisch wordt kan je een deel naar de Appendix verplaatsen. 

Probeer er een lopend verhaal van te maken.

Het is heel handig dit ook weer op te delen nav je deelvragen:

\subsubsection{RQ1}

\subsubsection{RQ2}
