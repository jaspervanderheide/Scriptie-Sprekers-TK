\section{Introduction}
\label{sec:intro}
Teksten van politieke partijen kunnen bruikbaar zijn voor het bepalen van ideologische positie van andere teksten, aangezien zij zowel een tekst leveren als ook een bekende ideologie. Deze informatie kan vervolgens toegepast worden bij andere teksten die wellicht ideologisch van aard zijn. Bijvoorbeeld, op basis van deze informatie kan men teksten uit kranten classificeren op basis van ideologie.\par

In dit onderzoek wordt getracht de lessen van bovengenoemde onderzoeken toe te passen op de Nederlandse politiek en te kijken in hoeverre er een voorspelbaarheid is in partij-affiliatie van Tweede Kamerleden aan de hand van hun handelingen. De onderzoeksvraag luidt daarom: "In hoeverre kan men politici in de Tweede Kamer aan de hand van hun handelingen classificeren wat betreft partij-affiliatie?"


\paragraph{Overview of thesis}
Hier geef je even kort weer wat in elke sectie staat.
