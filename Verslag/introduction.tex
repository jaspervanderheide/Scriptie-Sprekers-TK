\section{Introductie}
\label{sec:intro}
Teksten van politieke partijen kunnen dienen als bron voor het bepalen van ideologische positie van andere teksten, aangezien zij zowel tekst bevatten als ook een bekende ideologie in de vorm van een partij van de spreker; de partij-affiliatie. Het classificeren op basis van tekst kan inzichten geven over de relatie tussen ideologie en woordgebruik. Deze informatie kan vervolgens toegepast worden op andere teksten die wellicht ideologisch van aard zijn. Bijvoorbeeld kan men aan de hand van deze informatie teksten uit kranten classificeren op basis van ideologie \cite{DBLP:journals/corr/Biessmann16,Hirst_textto}.\par
In diverse landen zijn al onderzoeken gedaan naar het classificeren naar partij-affiliatie op basis van speeches in parlementen \cite{bhand,DBLP:journals/corr/Biessmann16,diermeier_godbout_yu_kaufmann_2012,Ferreira2016UsingTT,Hirst_textto,W14-2516,doi:10.1080/19331680802149608}. Met deze tekstclassificatie naar partij-affiliatie proberen onderzoekers zo goed mogelijk te classificeren. Daarnaast proberen ze vaak ook uit te vinden in hoeverre ideologie terug te vinden is in teksten van politici. De resultaten van de classificaties zijn in de meeste gevallen ruim boven de baseline. Hirst et al. \cite{Hirst_textto} vonden voor het Europese Parlement aanwijzingen dat dit het gevolg was van afstand van op de links-rechts as. Er zijn in deze onderzoeken ook redenen die suggereren dat dit niet alleen het gevolg is van ideologie. Zo suggereren de  resultaten van Hirst et al. op het Canadese parlement dat de partij-status (oppositie of regering) van invloed is op de classificatie. Daarnaast laat hun onderzoek naar Europese parlement ook zien dat partijnamen een grote invloed hebben op de classificatie.\par
Een onderzoek gericht op het Nederlandse parlement is niet gevonden. Ook beperken veel onderzoeken zich vaak tot één classificatiemethode.\par
Dit onderzoek richt zich daarom op meerdere classificatiemethoden. Daarnaast zal dit onderzoek zich richten op de Tweede Kamer. De onderzoeksvraag luidt daarom dus ook: "In hoeverre is classificatie naar partij-affiliatie aan de hand van spreekbeurten in de Tweede Kamer het gevolg van ideologie?"\par
Deze vraag wordt beantwoord door de antwoorden te vinden op de volgende deelvragen:
\begin{enumerate}
    \item Wat is de beste classificatiemethode voor classificatie naar partij-affiliatie in de Tweede Kamer en wat is het resultaat van dit model?
    \item In hoeverre is deze classificatie afhankelijk van achternamen van Kamerleden en partijnamen?
    \item In hoeverre wordt deze classificatie bepaald door partij-status (oppositie of regering)?
    \item In hoeverre wordt deze classificatie bepaald door positie op de links-rechts as?
    \item In hoeverre wordt deze classificatie bepaald door woordgebruik van sprekers?
\end{enumerate}
Hirst et al. \cite{Hirst_textto} vonden dat voor het Canadese parlement de partij-status van invloed was op de classificatie. In datzelfde onderzoek werd bij het Europese parlement geconstateerd dat ook partijnamen en positie op links-rechts as bepalend zijn. Ook levert dit onderzoek kritiek op een onderzoek van Diermeier et al. \cite{diermeier_godbout_yu_kaufmann_2012} waar getraind wordt op dezelfde sprekers als waar op getest wordt. Op basis van dit onderzoek is de hypothese dat al deze factoren van invloed zijn op de classificatie.\par
Voor de eerste deelvraag is Support Vector Machine, logistische regressie en Naive Bayes met verschillende parameters vergeleken aan de hand van \textit{accuracy} en $F_1$ score. Bij de tweede deelvraag is gekeken naar classificatie zonder achternamen van Kamerleden en partijnamen of met alleen achternamen van Kamerleden en partijnamen. De derde vraag bestaat uit drie experimenten. In de eerste is gekeken naar de hoeveelheid misclassificaties binnen regeringspartijen of binnen oppositiepartijen tegenover tussen een regeringspartij en een oppositiepartij. In de tweede is gekeken naar overlap in woordgebruik binnen regering. In de derde is gekeken naar verschil in scores als een partij gewisseld is van partij-status. Bij de vierde vraag is gekeken naar een verband tussen misclassificaties en afstand tussen twee partijen op de links-rechts as. Bij de vijfde vraag is de classificatie herhaald met Kamerleden verdeeld over training en test set.


\paragraph{Overzicht van scriptie}
Sectie 2 bevat vergelijkbare onderzoeken in andere parlementen. Sectie 3 bevat de methodologie van de verschillende deelvragen. Sectie 4 bevat de resultaten. Sectie 5 bevat de evaluatie van zowel de resultaten als de methodologie. Sectie 6 bevat ten slotte het antwoord op de onderzoeksvraag.
