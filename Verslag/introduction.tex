\section{Introductie}
\label{sec:intro}
Teksten van politieke partijen kunnen dienen als bron voor het bepalen van ideologische positie van andere teksten, aangezien zij zowel tekst bevatten als ook een bekende ideologie in de vorm van een partij-affiliatie. Het classificeren op basis van tekst kan inzichten geven over de relatie tussen ideologie en woordgebruik. Deze informatie kan vervolgens toegepast worden op andere teksten die wellicht ideologisch van aard zijn. Bijvoorbeeld kan men aan de hand van deze informatie teksten uit kranten classificeren op basis van ideologie \cite{DBLP:journals/corr/Biessmann16,Hirst_textto}.\par
In diverse landen zijn al onderzoeken gedaan naar het classificeren naar partij-affiliatie op basis van teksten van politici \cite{Ferreira2016UsingTT,DBLP:journals/corr/Biessmann16}. Met deze tekstclassificatie naar partij-affiliatie proberen onderzoekers uit te vinden in hoeverre ideologie terug te vinden is in teksten van politici. De resultaten van de tekstclassificaties zijn in alle gevallen ruim boven de baseline. Diverse onderzoeken wijzen daarentegen ook naar redenen dat dit niet alleen het gevolg is van ideologie. Zo suggereren de  resultaten van Hirst et al. \cite{Hirst_textto} dat de partij-status (oppositie tegenover regering) van invloed is op de classificatie. Daarnaast laat dit onderzoek ook zien dat partijnamen een grote invloed hebben op de classificatie.\par
Een onderzoek gericht op het Nederlandse parlement is niet gevonden. Ook beperken veel onderzoeken zich vaak tot één classificatiemethode.\par
Dit onderzoek richt zich daarom op meer classificatiemethoden en daarnaast specifiek op de Nederlandse politiek. De onderzoeksvraag luidt daarom dus ook: "In hoeverre is classificatie op basis van partij-affiliatie aan de hand van spreekbeurten in de Tweede Kamer het gevolg van ideologie?"\par
Deze vraag wordt beantwoord door de antwoorden te vinden op de volgende deelvragen:
\begin{enumerate}
    \item Wat is het beste classificatiemodel voor classificatie naar partij-affiliatie in de Tweede Kamer en wat is het resultaat van dit model?
    \item In hoeverre is deze classificatie afhankelijk van achternamen van Kamerleden en partijen?
    \item In hoeverre wordt deze classificatie bepaald door partij-status (d.w.z. oppositie of regering)?
    \item In hoeverre wordt deze classificatie bepaald door links/rechts verdeling?
    \item In hoeverre wordt deze classificatie bepaald door woordgebruik van sprekers?
\end{enumerate}
Voor de eerste deelvraag zullen Support Vector Machine, Logistische Regressie en Naive Bayes vergeleken worden aan de hand van \textit{accuracy} en $F_1$ score. Bij de tweede deelvraag wordt gekeken naar classificatie zonder achternamen van Kamerleden en partijen of met alleen achternamen van Kamerleden en partijen. De derde vraag bestaat uit meerdere experimenten, waarin gekeken zal worden naar of de misclassificaties binnen coalitie of oppositie groter zijn dan daartussen, en of er tussen die groepen verschillen zitten in de confusion matrix. 


\paragraph{Overzicht van scriptie}
Sectie 2 bevat gerelateerd werk, met name vergelijkbare onderzoeken in andere landen. Sectie 3 bevat de methodologie van de verschillende deelvragen. Sectie 4 bevat de resultaten. Sectie 5 bevat de evaluatie van zowel de resultaten als de gehanteerde methodologie. Sectie 6 bevat ten slotte het antwoord op de onderzoeksvraag.
