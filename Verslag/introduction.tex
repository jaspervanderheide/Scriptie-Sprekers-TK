\section{Introduction}
\label{sec:intro}
Teksten van politieke partijen kunnen bruikbaar zijn voor het bepalen van ideologische positie van andere teksten, aangezien zij zowel een tekst leveren als ook een bekende ideologie. Deze informatie kan vervolgens toegepast worden bij andere teksten die wellicht ideologisch van aard zijn. Bijvoorbeeld, op basis van deze informatie kan men teksten uit kranten classificeren op basis van ideologie.\par
In diverse landen zijn al verschillende onderzoeken gedaan naar het classificeren van partij-affiliatie op basis van teksten van politici.\cite{Ferreira2016UsingTT} Mede omdat elk land een andere politiek stelsel en cultuur heeft, verschillen de resultaten. Daarnaast gebruikt elk onderzoek ook een andere methode voor het classificeren. \par
Een onderzoek gericht op het Nederlandse parlement ontbreekt hierbij nog. Daarnaast focust elk onderzoek tot nu toe op een beperkte aantal methoden, dus geen brede analyse van de mogelijke methoden. \par
Dit onderzoek richt zich daarom op een breder scala aan mogelijke methoden en daarnaast ook specifiek op de Nederlandse politiek. De onderzoeksvraag luidt daarom dus ook: "Wat is het beste classificatiemodel voor het classificeren van sprekers in de Tweede Kamer op basis van partij-affiliatie?"\par
In dit onderzoek zal eerst gekeken worden naar welke methoden gangbaar zijn in vergelijkbare onderzoeken, maar ook naar welke methoden nieuw en potentieel zijn. Deze worden vervolgens geëvalueerd en vergeleken, in de hoop hiermee de onderzoeksvraag te kunnen beantwoorden.


\paragraph{Overview of thesis}
In sectie 2 zullen vergelijkbare onderzoeken uit andere landen besproken worden. In sectie 3 zal vervolgens de wijze waarop de verschillende classificatiemethoden gebruikt zijn als ook geëvalueerd zijn besproken worden. In sectie 4 zullen vervolgens de resultaten weergegeven worden. In sectie 5 zullen dan een evaluatie plaatsvinden van zowel de resultaten als de gehanteerde methodologie. In sectie 6 wordt dan ten slotte de onderzoeksvraag beantwoord.
