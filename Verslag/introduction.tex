\section{Introductie}
\label{sec:intro}
Teksten van politieke partijen kunnen dienen als bron voor het bepalen van ideologische positie van andere teksten, aangezien zij zowel tekst hebben als ook een bekende ideologie. Deze informatie kan vervolgens toegepast worden bij andere teksten die wellicht ideologisch van aard zijn. Bijvoorbeeld, aan de hand van deze informatie kan men teksten uit kranten classificeren op basis van ideologie \cite{DBLP:journals/corr/Biessmann16,Hirst_textto}.\par
In diverse landen zijn al verschillende onderzoeken gedaan naar het classificeren van partij-affiliatie op basis van teksten van politici\cite{Ferreira2016UsingTT,DBLP:journals/corr/Biessmann16}. Met deze tekst classificatie proberen onderzoekers ideologie uit te vinden in hoeverre ideologie terug te vinden is in teksten van politici. Mede omdat elk land een ander stelsel, taal en cultuur heeft, verschillen de resultaten. Elk onderzoek gebruikt ook een andere methode voor het classificeren. Daarnaast vindt het onderzoek van Hirst et al. \cite{Hirst_textto} dat deze classificatie minder het gevolg is van ideologie maar meer van regering tegenover oppositie. Deze onderzoeken worden besproken in het gerelateerd werk.\par
Een onderzoek gericht op het Nederlandse parlement is niet gevonden. Ook beperken veel onderzoeken zich vaak tot één classificatiemethode.\par
Dit onderzoek richt zich daarom op een breder scala aan mogelijke methoden en daarnaast specifiek op de Nederlandse politiek. De onderzoeksvraag luidt daarom dus ook: "In hoeverre is classificatie op basis van partij-affiliatie aan de hand van spreekbeurten in de Tweede Kamer het gevolg van ideologie?"\par
Deze vraag wordt beantwoord door de antwoorden te vinden op de volgende deelvragen:
\begin{enumerate}
    \item Wat is het beste classificatiemodel voor classificatie van partij-affiliatie in de Tweede Kamer en wat is het resultaat van dit model?
    \item In hoeverre is deze classificatie afhankelijk van partijnamen en namen van Kamerleden?
    \item In hoeverre wordt deze classificatie bepaald door partij-status (d.w.z. oppositie of regering)?
    \item In hoeverre wordt deze classificatie bepaalt door links/rechts verdeling?
\end{enumerate}
Daarom zal eerst gekeken worden naar classificatiemethoden en resultaten in vergelijkbare onderzoeken. Van deze classificatiemethoden zullen een aantal toegepast worden op teksten van de Tweede Kamer. Vervolgens zal door middel van de overige deelvragen bepaald worden in hoeverre dit een reflectie is van ideologie.


\paragraph{Overzicht van scriptie}
Sectie 2 bevat gerelateerd werk, met name vergelijkbare onderzoeken in andere landen. Sectie 3 bevat de methodologie van de verschillende deelvragen. Sectie 4 bevat de resultaten. Sectie 5 bevat de evaluatie van zowel de resultaten als de gehanteerde methodologie. Sectie 6 bevat ten slotte het antwoord op de onderzoeksvraag.
