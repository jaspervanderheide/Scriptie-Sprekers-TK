\section{Introduction}
\label{sec:intro}
Teksten van politieke partijen kunnen bruikbaar zijn voor het bepalen van ideologische positie van andere teksten, aangezien zij zowel een tekst leveren als ook een bekende ideologie. Deze informatie kan vervolgens toegepast worden bij andere teksten die wellicht ideologisch van aard zijn. Bijvoorbeeld, op basis van deze informatie kan men teksten uit kranten classificeren op basis van ideologie.\\

Diermeier et al. (2007) deed onderzoek naar het classificeren op basis van ideologische positie in de Amerikaanse Senaat (101e tot en met 108e Congres). Dit onderzoek wist de ideologie van de senatoren te voorspellen met een 92 procent nauwkeurigheid.\\
Als een vervolg op dit onderzoek deed Graeme Hirst et al. (2010) een vergelijkbaar onderzoek naar zowel het Canadese als het Europese Parlement. \cite{Hirst_textto} In dit onderzoek maken zij gebruik van support-vector machines. In tegenstelling tot het onderzoek van Diermeier et al. (2007), vinden zij minder dat de woorden van de sprekers een uiting zijn van ideologie. Daarentegen vinden zij wel een grotere invloed van oppositie tegenover regering in de woorden van de sprekers.\\

In dit onderzoek wordt getracht de lessen van bovengenoemde onderzoeken toe te passen op de Nederlandse politiek en te kijken in hoeverre er een voorspelbaarheid is in partij-affiliatie van Tweede Kamerleden aan de hand van hun handelingen. De onderzoeksvraag luidt daarom: "In hoeverre kan men politici in de Tweede Kamer aan de hand van hun handelingen classificeren wat betreft partij-affiliatie?"


\paragraph{Overview of thesis}
Hier geef je even kort weer wat in elke sectie staat.
