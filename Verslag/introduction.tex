\section{Introductie}
\label{sec:intro}
Teksten van politieke partijen kunnen bruikbaar zijn voor het bepalen van ideologische positie van andere teksten, aangezien zij zowel een tekst leveren als ook een bekende ideologie. Deze informatie kan vervolgens toegepast worden bij andere teksten die wellicht ideologisch van aard zijn. Bijvoorbeeld, op basis van deze informatie kan men teksten uit kranten classificeren op basis van ideologie.\cite{DBLP:journals/corr/Biessmann16}\cite{Hirst_textto}\par
In diverse landen zijn al verschillende onderzoeken gedaan naar het classificeren van partij-affiliatie op basis van teksten van politici.\cite{Ferreira2016UsingTT}\cite{DBLP:journals/corr/Biessmann16} Mede omdat elk land een andere politiek stelsel en cultuur heeft, verschillen de resultaten. Daarnaast gebruikt elk onderzoek ook een andere methode voor het classificeren. \par
Een onderzoek gericht op het Nederlandse parlement ontbreekt hierbij nog. \par
Dit onderzoek richt zich daarom op een breder scala aan mogelijke methoden en daarnaast ook specifiek op de Nederlandse politiek. De onderzoeksvraag luidt daarom dus ook: "In hoeverre is classificatie op basis van partij-affiliatie aan de hand van spreekbeurten in de Tweede Kamer het gevolg van ideologie?"\par
Deze vraag wordt beantwoord door de antwoorden te vinden op de volgende deelvragen:
\begin{enumerate}
    \item Wat is het beste classificatiemodel voor classificatie van partij-affiliatie in de Tweede Kamer en wat is het resultaat van dit model?
    \item In hoeverre is deze classificatie afhankelijk van partijnamen en namen van Kamerleden?
    \item In hoeverre wordt deze classificatie bepaald door of een partij in regering of oppositie zit?
    \item In hoeverre is deze classificatie afhankelijk van of een partij rechts of links is?
\end{enumerate}
%In dit onderzoek zal eerst gekeken worden naar welke methoden gangbaar zijn in vergelijkbare onderzoeken, maar ook naar welke methoden nieuw en potentieel zijn. Deze worden vervolgens geëvalueerd en vergeleken, in de hoop hiermee de onderzoeksvraag te kunnen beantwoorden.


\paragraph{Overzicht van scriptie}
In sectie 2 zal gerelateerd werk besproken worden, met name vergelijkbare onderzoeken uit andere landen. In sectie 3 zal de methodologie van de verschillende deelvragen behandeld worden. In sectie 4 zullen vervolgens de resultaten weergegeven worden. In sectie 5 zal een evaluatie plaatsvinden van zowel de resultaten als de gehanteerde methodologie. In sectie 6 wordt ten slotte de onderzoeksvraag beantwoord.
