\section{Related Work}
\label{sec:rel}

Diermeier et al. deed onderzoek naar het classificeren op basis van ideologische positie in de Amerikaanse Senaat (101e tot en met 108e Congres) \cite{diermeier_godbout_yu_kaufmann_2012}. Dit onderzoek wist de ideologie van de senatoren te voorspellen met een 92 procent nauwkeurigheid.\par
Als een vervolg op dit onderzoek deden Graeme Hirst et al. een vergelijkbaar onderzoek naar zowel het Canadese als het Europese Parlement\cite{Hirst_textto}. In dit onderzoek maken zij gebruik van support-vector machines. In tegenstelling tot het onderzoek van Diermeier et al., vinden zij minder dat de woorden van de sprekers een uiting zijn van ideologie. Daarentegen vinden zij wel een grotere invloed van oppositie tegenover regering in de woorden van de sprekers.\par
Ferreira classificeerde teksten van politici in Portugese parlement op basis van geslacht, leeftijdsgroep, affiliatie en oriëntatie\cite{Ferreira2016UsingTT}. Hij gebruikte hiervoor MIRA classificatie en logische regressie in combinatie met Group Lasso regularisatie. In alle gevallen kon men aan de hand van logische regressie en group Lasso regularisatie een F1-score van 0.87 of hoger bereiken.\par


\subsection{RQ1}

\subsection{RQ2}
