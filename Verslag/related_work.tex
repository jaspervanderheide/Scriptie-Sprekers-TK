\section{Gerelateerd werk}
\label{sec:rel}




\subsection{Deelvraag 1}
\label{sec:Deelvraag1}
Diermeier et al. deden onderzoek naar het classificeren op basis van ideologische positie in de Amerikaanse Senaat\cite{diermeier_godbout_yu_kaufmann_2012}. Ze trainden hun classicatie op de speeches van de 25 meest liberale en de 25 meest conservatieve senatoren van het 101e tot en met het 107e congres en testten op de 25 meest liberale en de 25 meest conservatieve senatoren van het 108e congres. Later in het onderzoek vergeleken ze ook de 25 gematigd conservatieve  en de 25 gematigd liberale senatoren. Voor classificatie maakten ze gebruik van support vector machines. Verder maakten ze gebruik van TF-IDF met een minimale woordfrequentie van 50 en een documentfrequentie van 10, \textit{Part-Of-Speech tagging} en werden alle eigennamen verwijderd. Dit onderzoek wist de ideologie van de senatoren te voorspellen met een 94 procent nauwkeurigheid voor de classificatie van de extremen, maar slechts een 52 procent nauwkeurigheid voor de classificatie van de gematigde senatoren.\par
Als een vervolg op dit onderzoek deden Graeme Hirst et al. een vergelijkbaar onderzoek naar zowel het Canadese als het Europese Parlement\cite{Hirst_textto}. In dit onderzoek maken zij gebruik van support-vector machines. In tegenstelling tot het onderzoek van Diermeier et al., vinden zij minder dat de woorden van de sprekers een uiting zijn van ideologie. Daarentegen vinden zij wel een grotere invloed van oppositie tegenover regering in de woorden van de sprekers.\par
Ferreira probeerde interventies van politici te classificeren op basis van geslacht, leeftijdsgroep, partij-affiliatie en ori\"{e}ntatie\cite{Ferreira2016UsingTT} In dit onderzoek maakt hij gebruik van twee classificatiemethoden, Logistische regressie en MIRA. Logistische regressie werd aangevuld met \textit{group Lasso} regularisatie. Voor wegingen van woorden werd gebruikt gemaakt van woordfrequentie, TF-IDF, $\Delta$-TF-IDF, $\Delta$-BM-25. Daarnaast wordt er gebruik gemaakt van woordclustering, \textit{Concise Semantic Analysis} en stylometrische eigenschappen. Op \textit{Part-Of-Speech tagging} na hadden stylometrische eigenschappen een duidelijke negatieve invloed op de classificatie. In alle classificaties kon men aan de hand van logistische regressie en \textit{group Lasso} regularisatie een F1-score van 0.87 of hoger bereiken.\par

\subsection{Deelvraag 3}
Graeme Hirst et al. vonden in hun onderzoek dat de classificatie van spreker in het Canadese parlement op basis van partij-affiliatie meer zegt over de status van de partij (regering of oppositie).\cite{Hirst_textto} Zo vergeleken zij de top tien karakteristieke woorden van de liberalen en conservatieven in het 36e parlement (liberalen in de regering) en het 39e parlement (conservatieven in de regering. Hier vonden zij dat vier van de tien woorden van de liberalen (regering) in het 36e parlement bij het 39e parlement bij de conservatieven (regering) te vinden waren. Andersom gebeurde hetzelfde met één van de tien woorden van de conservatieven (oppositie) in het 36e parlement naar liberalen (oppositie) in het 39e parlement.\par
In hetzelfde onderzoek trainden ze ook hun classifiers op het ene parlement en testten deze op het andere parlement. Hierbij vonden zij in beide gevallen een nauwkeurigheid ver onder de baseline. Daarnaast deden ze ook nog een classificatie op de sprekers die in beide parlementen zaten en een andere classificatie op sprekers die niet in beide parlementen zaten. Bij de eerste classificatie vonden ze nauwkeurigheden rond de baseline, terwijl in de tweede situatie  nauwkeurigheden gevonden werden ver boven de baseline. \par
Deze resultaten leidden de onderzoekers tot de conclusie dat de classificatie voornamelijk het gevolg is van de status van de partij en minder van ideologie.
