\section{Related Work}
\label{sec:rel}

Diermeier et al. deed onderzoek naar het classificeren op basis van ideologische positie in de Amerikaanse Senaat (101e tot en met 108e Congres) \cite{diermeier_godbout_yu_kaufmann_2012}. Dit onderzoek wist de ideologie van de senatoren te voorspellen met een 92 procent nauwkeurigheid.\par
Als een vervolg op dit onderzoek deden Graeme Hirst et al. een vergelijkbaar onderzoek naar zowel het Canadese als het Europese Parlement\cite{Hirst_textto}. In dit onderzoek maken zij gebruik van support-vector machines. In tegenstelling tot het onderzoek van Diermeier et al., vinden zij minder dat de woorden van de sprekers een uiting zijn van ideologie. Daarentegen vinden zij wel een grotere invloed van oppositie tegenover regering in de woorden van de sprekers.\par
Ferreira probeerde interventies van politici te classificeren op basis van geslacht, leeftijdsgroep, partij-affiliatie en ori\"{e}ntatie\cite{Ferreira2016UsingTT} In dit onderzoek maakt hij gebruik van twee classificatiemethoden, Logistische regressie en MIRA. Logistische regressie werd aangevuld met \textit{group Lasso} regularisatie. Voor wegingen van woorden werd gebruikt gemaakt van woordfrequentie, TF-IDF, $\Delta$-TF-IDF, $\Delta$-BM-25. Daarnaast wordt er gebruik gemaakt van woordclustering, \textit{Concise Semantic Analysis} en stylometrische eigenschappen, waaronder \textit{Part-Of-Speech tagging}. In alle classificaties kon men aan de hand van logistische regressie en \textit{group Lasso} regularisatie een F1-score van 0.87 of hoger bereiken.\par


\subsection{RQ1}

\subsection{RQ2}
