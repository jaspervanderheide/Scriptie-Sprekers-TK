\section{Gerelateerd werk}
\label{sec:rel}

Diermeier et al. deden onderzoek naar het classificeren op basis van ideologische positie in de Amerikaanse Senaat\cite{diermeier_godbout_yu_kaufmann_2012}. Ze trainden hun classificatie op de speeches van de 25 meest liberale en de 25 meest conservatieve senatoren van het 101e tot en met het 107e congres en testten op de 25 meest liberale en de 25 meest conservatieve senatoren van het 108e congres. Later in het onderzoek vergeleken ze ook de 25 gematigd conservatieve  en de 25 gematigd liberale senatoren.  Dit onderzoek wist de ideologie van de senatoren te voorspellen met een 94 procent nauwkeurigheid voor de classificatie van de extremen, maar slechts een 52 procent nauwkeurigheid voor de classificatie van de gematigde senatoren.\par

Als een vervolg op dit onderzoek deden Graeme Hirst et al. een vergelijkbaar onderzoek naar zowel het Canadese Parlement \cite{Hirst_textto}. Hierbij werd zowel gekeken naar de Engelse als Franse teksten. Een document werd hier gezien als de samenvoeging van alle spreekbeurten van een spreker. Afhankelijk van taal en dataset, vinden zij in dit onderzoek nauwkeurigheden van 83.2 procent en hoger. In tegenstelling tot het onderzoek van Diermeier et al., vinden zij minder dat de woorden van de sprekers een uiting zijn van ideologie. \par

Het onderzoek van Bhand et al. richtte zich op het classificeren van leden van het Amerikaanse congres in 2005, op basis van affiliatie (Republikeins of Democratisch)\cite{bhand}. Zij vonden hiervoor uiteindelijk een $F_1$ score van 0.684647.\par

Ferreira probeerde interventies van politici te classificeren op basis van geslacht, leeftijdsgroep, partij-affiliatie en ori\"{e}ntatie in het Portugese parlement \cite{Ferreira2016UsingTT}. In alle classificaties kon men een $F_1$ score van 0.87 of hoger bereiken.\par
In het onderzoek van Høyland et al. werd een classificatiemodel voor partij-affiliatie op basis van teksten getraind op het vijfde Europese Parlement (1999-2004) en getest op het zesde Europese Parlement\cite{W14-2516}. Hier verkregen zij een \textit{macro average} $F_1$ score van 0.464.\par

\subsection{Classificatiemethoden}
\label{sec:Deelvraag1}
In het onderzoek van Diermeier et al. werd gebruik gemaakt van support vector machines. Verder maakten ze gebruik van \textit{tf-idf} met een minimale woordfrequentie van 50 en een documentfrequentie van 10, \textit{Part-Of-Speech tagging} en werden alle eigennamen verwijderd.\par
In het onderzoek van Graeme Hirst et al. maakten ze gebruik van support vector machines\cite{Hirst_textto}. Ze experimenteerden met verschillende vormen van pre-processing, inclusief stemmen en het verwijderen van woorden op basis van te hoge of te lage frequentie. Deze variaties maakten in hun onderzoek geen grote verschillen en uiteindelijk is gekozen voor het niet stemmen, het weglaten van woorden die in minder dan vijf documenten voorkomen en resultaten van zowel met als zonder de top 500 meest frequente woorden. Daarnaast werd geëxperimenteerd met vier wegingen voor woorden: \textit{boolean}, \textit{tf}, \textit{tf-norm} en \textit{tf-idf}, waarvan \textit{tf-idf} het beste resultaat opleverde. \par
In het onderzoek van Bhand et al. gebruikten ze verschillende n-grams, inclusief verschillende manieren van \textit{smoothing}\cite{bhand}. Zij gebruikte als weging altijd de aanwezigheid van een woord. Als classificatiemodellen experimenteerden ze support vector machines en naive bayes classificatie. Voor het selecteren van \textit{features} experimenteerden ze met een simpele minimale frequentie en het gebruik van een top aantal woorden op basis van mutual information. Uiteindelijk was het beste model bij hen een met support vector machine, met uni- en bigrams, gekozen op basis van mutual information.\par
In het onderzoek van Ferreira werd gebruik gemaakt van twee classificatiemethoden: Logistische regressie en MIRA\cite{Ferreira2016UsingTT}. Logistische regressie werd aangevuld met \textit{group Lasso} regularisatie. Voor wegingen van woorden werd geëxperimenteerd met \textit{tf}, \textit{tf-idf}, \textit{$\Delta$-tf-idf} en \textit{$\Delta$-BM-25}. Daarnaast wordt er gebruik gemaakt van woordclustering, \textit{Concise Semantic Analysis} en stylometrische eigenschappen. Op \textit{Part-Of-Speech tagging} na hadden stylometrische eigenschappen een duidelijke negatieve invloed op de classificatie.\par
In het onderzoek van Høyland et al. werd gebruik gemaakt van een multi class support vector machine\cite{W14-2516}. Als beste waarde voor de regularisatieterm, de C-parameter, vonden zij 0.8. Daarnaast gebruikten zij \textit{dependency disambiguated
stems} wat bij hen een $F_1$ score van twee procent hoger opleverden dan normale stemming.\par

\subsection{Invloed van partijnamen of sprekersnamen}
In het onderzoek van Diermeier et al. zijn alle namen weggelaten, omdat deze volgens hen de classificatie te makkelijk zouden maken \cite{diermeier_godbout_yu_kaufmann_2012}. Hirst et al. vinden inderdaad dat partijnamen (en het weglaten daarvan) bij het Europees Parlement een grote invloed hebben op de classificatie \cite{Hirst_textto}. Bij het Europees Parlement zien zij met name het gebruik van de eigen partijnaam door een spreker, terwijl zij in het Canadese parlement vooral zien dat de naam van de andere partij gebruikt wordt door een spreker.

\subsection{Invloed van oppositie of regering}
Graeme Hirst et al. vonden in hun onderzoek dat de classificatie van spreker in het Canadese parlement op basis van partij-affiliatie meer zegt over de status van de partij (regering of oppositie).\cite{Hirst_textto} Zo vergeleken zij de top tien karakteristieke woorden van de liberalen en conservatieven in het 36e parlement (liberalen in de regering) en het 39e parlement (conservatieven in de regering. Hier vonden zij dat vier van de tien woorden van de liberalen (regering) in het 36e parlement bij het 39e parlement bij de conservatieven (regering) te vinden waren. Andersom gebeurde hetzelfde met één van de tien woorden van de conservatieven (oppositie) in het 36e parlement naar liberalen (oppositie) in het 39e parlement.\par
In hetzelfde onderzoek trainden ze ook hun classifiers op het ene parlement en testten deze op het andere parlement. Hierbij vonden zij in beide gevallen een nauwkeurigheid ver onder de baseline. Daarnaast deden ze ook nog een classificatie op de sprekers die in beide parlementen zaten en een andere classificatie op sprekers die niet in beide parlementen zaten. Bij de eerste classificatie vonden ze nauwkeurigheden rond de baseline, terwijl in de tweede situatie nauwkeurigheden gevonden werden ver boven de baseline. \par
Deze resultaten leidden de onderzoekers tot de conclusie dat de classificatie voornamelijk het gevolg is van de status van de partij en minder van ideologie.\par



% https://nlp.stanford.edu/courses/cs224n/2009/fp/7.pdf