\section{Evaluatie}
\label{sec:eva}

\subsection{Resultaten}

\subsubsection{Deelvraag 1}
Het beste resultaat werd bereikt met SVM gebruikmakend van \textit{stochastic gradient descent learning} en Ridge regularisatie.\par

Figuur \ref{fig:confusionmatrix} laat zien waar de fouten in deze classificatie zitten. De meest karakteristieke features per partij zijn te zien in figuur \ref{tab:MostImportantWords}. Beide resultaten zijn op basis van één classificatie.


\begin{figure}[H]
  \centering
    \includegraphics[width=0.6\paperwidth]{Verslag/confusionmatrix.png}
\caption{Confusion matrix van beste classificatie.}
\label{fig:confusionmatrix}
\end{figure}

\begin{table}[H]
\caption{Meest relevante woorden per partij op basis van beste classificatie.} 
\label{tab:MostImportantWords} 
\centering
\hspace*{-1in}
\begin{tabular}{lllll}
\toprule
          50PLUS &               CDA &         ChristenUnie &                  D66 &         GroenLinks \\
\midrule
          50plus &               cda &      de christenunie &                  d66 &         groenlinks \\
    het lid krol &           het cda &         christenunie &         mijn fractie &   lid van tongeren \\
        lid krol &            de cda &        lid dik faber &  leden van veldhoven &   het lid voortman \\
   lid krol naar &       cda fractie &              lid dik &        van veldhoven &  lid voortman naar \\
   krol naar mij &    de cda fractie &          het lid dik &            veldhoven &       lid voortman \\
       krol naar &   het lid omtzigt &            dik faber &    lid van veldhoven &       van tongeren \\
            krol &       lid omtzigt &     leden voordewind &       lid van meenen &           tongeren \\
      van 50plus &  lid omtzigt naar &  de leden voordewind &               d66 is &           voortman \\
 gepensioneerden &  omtzigt naar mij &                faber &    de leden bergkamp &  de leden voortman \\
         ouderen &      omtzigt naar &                  dik &       leden bergkamp &     leden voortman \\
\bottomrule
\end{tabular}
 
\end{table} 
\addtocounter{table}{-1} 
\begin{table}[H]
\caption{Meest relevante woorden per partij op basis van beste classificatie. \emph{(Vervolg)}} 
\centering
\hspace*{-1in}
\begin{tabular}{llllll}
\toprule
              PVV &             PvdA &              PvdD &              SGP &             SP &             VVD \\
\midrule
              pvv &          de pvda &      lid ouwehand &              sgp &             sp &          de vvd \\
           de pvv &             pvda &  lid ouwehand nar &           de sgp &          de sp &             vvd \\
      islamitisch &    de partij van &  het lid ouwehand &      sgp fractie &   lid van gerv &  de vvd fractie \\
        lid graus &    van de arbeid &      ouwehand nar &   de sgp fractie &     sp fractie &     vvd fractie \\
    het lid graus &        de arbeid &  ouwehand nar mij &     led dijkgraf &  de sp fractie &       de vvd is \\
    lid graus nar &    partij van de &          ouwehand &  de led dijkgraf &   van gerv nar &          vvd is \\
          miljard &       partij van &       vor de dier &      led van der &       gerv nar &      vor de vvd \\
    graus nar mij &     pvda fractie &           de dier &  de led bisschop &   gerv nar mij &      wat de vvd \\
        graus nar &           arbeid &              dier &     led bisschop &           gerv &     vvd betreft \\
 madlener nar mij &  de pvda fractie &     de partij vor &           sgp is &       van gerv &  de vvd betreft \\
\bottomrule
\end{tabular}
 
\end{table}

\subsubsection{Deelvraag 2}
In figuur \ref{tab:MostImportantWords} was al te zien dat de meest karakteristieke woorden voornamelijk bestaan uit partijnamen en namen van Kamerleden. 



In figuur \ref{tab:MostImportantWordsWithoutNames} is vervolgens te zien welke woorden het meest karakteristiek zijn per partij, als partijnamen namen van Kamerleden vervangen zijn door een generieke placeholder.
\begin{table}[H] 
\caption{Meest relevante woorden per partij op basis van classificatie zonder partijnamen of namen van Kamerleden.} 
\label{tab:MostImportantWordsWithoutNames} 
\centering
\hspace*{-1in}
\begin{tabular}{lllll}
\toprule
                 50PLUS &             CDA &        ChristenUnie &           D66 &              GroenLinks \\
\midrule
                ouderen &  PARTIJ fractie &       vluchtelingen &  mijn fractie &                     zou \\
        gepensioneerden &        inwoners &           inderdaad &          mijn &       kamer hierover te \\
               plussers &          PARTIJ &        mensenhandel &    natuurlijk &        persoonsgebonden \\
 koopkrachtontwikkeling &        regering &              zullen &       fractie &            in elk geval \\
                 oudere &     de regering &            gezinnen &   het kabinet &               elk geval \\
                     50 &            echt &  voedselverspilling &       vandaag &                  in elk \\
            50 plussers &         fractie &          constateer &  buitengewoon &             hierover te \\
              werkenden &            hier &       ik constateer &    belangrijk &     belastingontwijking \\
        overwegende dat &             wij &          onder meer &      minister &             regering om \\
            overwegende &            zeer &         begeleiding &       kabinet &  hierover te informeren \\
\bottomrule
\end{tabular}
 
\end{table} 
\addtocounter{table}{-1} 
\begin{table}[H] 
\caption{Meest relevante woorden per partij op basis van classificatie zonder partijnamen of namen van Kamerleden. \emph{(Vervolg)}} 
\centering
\hspace*{-0.6in}
\begin{tabular}{llllll}
\toprule
              PVV &         PvdA &              PvdD &                    SGP &            SP &            VVD \\
\midrule
      islamitisch &       kinder &              dier &  mevrouw de voorzitter &          zegt &     volgen mij \\
        nederland &       jonger &          de natur &             mevrouw de &       huurder &  yyyyy fractie \\
          miljard &  mijn partij &            de bio &               dank zer &          mens &         volgen \\
          brussel &      gezamen &     bio industrie &                   punt &     de bevolk &          yyyyy \\
 belastingbetaler &        beter &     constater dat &           eenverdiener &        bevolk &         veilig \\
               al &       tevred &               bio &                    wel &        bezuin &       regelgev \\
            islam &      collega &  de bio industrie &            bewindslied &    bestuurder &     essentieel \\
         de islam &          die &            milieu &               allerlei &  bureaucratie &       aangegev \\
          allemal &          sam &         industrie &                     je &          toch &     ondernemer \\
          mijnher &    circulair &         constater &      vor de beantwoord &    mening dat &          kader \\
\bottomrule
\end{tabular}
 
\end{table}

\subsection{Discussie}
\subsubsection{Deelvraag 1}
Dit onderzoek heeft zich beperkt tot methoden genoemd in vergelijkbare onderzoeken én waarvan de implementatie beschikbaar is in Python. Een aantal methoden die in gerelateerde literatuur leidden tot goede classificaties zijn daarom niet getest. Ook nieuwe methoden die nog niet gebruikt zijn in een vergelijkbaar onderzoek voor politieke tekst classificatie zijn daarom niet getest. Omdat dus niet alle opties getest zijn, kan geen uitsluitsel gegeven worden dat dit daadwerkelijk het classificatiemodel is. Voor vervolgonderzoek kan daarom gekeken worden naar meer verschillende methoden.\par

\subsubsection{Deelvraag 1}


\subsubsection{Deelvraag 4}
Er zijn verschillende visies op links en rechts, en de indeling van de partijen, ook buiten de twee methoden gekozen in dit onderzoek.\par