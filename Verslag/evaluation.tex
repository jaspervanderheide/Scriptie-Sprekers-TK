
\section{Resultaten}

\subsection{DV1: Beste classificatiemethode}
Het beste resultaat werd bereikt met Support Vector Machines gebruikmakend van \textit{stochastic gradient descent learning} en Elasticnet regularisatie. De features waren hierbij gestemd, met unigrams, bigrams en trigrams. Geen features zijn hierin weggelaten door minimale of maximale documentfrequenties. Het maximum aantal iteraties was 5 voor de grid search, maar alle resultaten zijn op basis van 100.\par
Tabel \ref{tab:classrapport} laat de scores zien per partij met het aantal documenten in de test set. De $F_1$ scores per partij liggen tussen de 0.7 en 0.9. De one-issuepartijen, 50PLUS, PVV en PvdD, als ook de SGP hebben hoge scores, terwijl de coalitiepartijen, VVD en PvdA, lagere scores hebben. Figuur \ref{fig:confusionmatrix} laat zien waar de fouten in deze classificatie zitten. De meest karakteristieke features per partij zijn te zien in tabel \ref{tab:MostImportantWords}. Met meest karakteristiek worden de n-grams bedoeld die de hoogste coëfficiënt hebben in de classificatie en die dus relatief het meeste invloed hebben. Hierin is te zien dat vrijwel alle n-grams achternamen van Kamerleden of partijnamen bevatten.\par

\begin{table}[H]
\caption{Classificatie scores per partij van beste classificatiemethode (SVM). Gemiddelde van vijf splitsingen van training en test set. Maximum aantal iteraties is 100.}
\label{tab:classrapport}
\centering
\begin{tabular}{lrrrr}
\toprule
{} &  Precision &  Recall &  F1\_score &  Documenten \\
Partij       &            &         &           &             \\
\midrule
50PLUS       &      0.920 &   0.852 &     0.884 &        77.0 \\
   CDA       &      0.760 &   0.774 &     0.768 &       378.8 \\
ChristenUnie &      0.806 &   0.780 &     0.788 &       210.4 \\
   D66       &      0.806 &   0.704 &     0.746 &       401.6 \\
  GroenLinks &      0.824 &   0.752 &     0.784 &       215.8 \\
   PVV       &      0.752 &   0.850 &     0.800 &       344.4 \\
  PvdA       &      0.696 &   0.658 &     0.670 &       351.0 \\
  PvdD       &      0.868 &   0.842 &     0.852 &        86.8 \\
   SGP       &      0.782 &   0.802 &     0.788 &       133.0 \\
    SP       &      0.756 &   0.816 &     0.784 &       457.8 \\
   VVD       &      0.718 &   0.696 &     0.704 &       323.4 \\
 avg / total &      0.768 &   0.762 &     0.762 &      2980.0 \\
\bottomrule
\end{tabular}

\end{table}


\begin{figure}[H]
  \centering
    \includegraphics[width=0.60\paperwidth]{Verslag/Tables/confusionmatrix.png}
\caption{Confusion matrix van beste classificatie. Gemiddelde van vijf splitsingen van training en test set.}
\label{fig:confusionmatrix}
\end{figure}



\begin{table}[H]
\caption{Meest karakteristieke n-grams per partij op basis van beste classificatie gedurende kabinet-Rutte II. N-grams die niet achternamen van Kamerleden of partijnamen bevatten, zijn dikgedrukt. }
\label{tab:MostImportantWords} 
\centering
\hspace*{-1in}
\begin{tabular}{lllll}
\toprule
          50PLUS &               CDA &         ChristenUnie &                  D66 &         GroenLinks \\
\midrule
          50plus &               cda &      de christenunie &                  d66 &         groenlinks \\
    het lid krol &           het cda &         christenunie &         mijn fractie &   lid van tongeren \\
        lid krol &            de cda &        lid dik faber &  leden van veldhoven &   het lid voortman \\
   lid krol naar &       cda fractie &              lid dik &        van veldhoven &  lid voortman naar \\
   krol naar mij &    de cda fractie &          het lid dik &            veldhoven &       lid voortman \\
       krol naar &   het lid omtzigt &            dik faber &    lid van veldhoven &       van tongeren \\
            krol &       lid omtzigt &     leden voordewind &       lid van meenen &           tongeren \\
      van 50plus &  lid omtzigt naar &  de leden voordewind &               d66 is &           voortman \\
 gepensioneerden &  omtzigt naar mij &                faber &    de leden bergkamp &  de leden voortman \\
         ouderen &      omtzigt naar &                  dik &       leden bergkamp &     leden voortman \\
\bottomrule
\end{tabular}
 
\end{table} 
\addtocounter{table}{-1} 
\begin{table}[H]
\caption{Meest karakteristieke n-grams per partij op basis van beste classificatie gedurende kabinet-Rutte II. N-grams die niet achternamen van Kamerleden of partijnamen bevatten, zijn dikgedrukt.  \emph{(Vervolg)}} 
\centering
\hspace*{-1in}
\begin{tabular}{llllll}
\toprule
              PVV &             PvdA &              PvdD &              SGP &             SP &             VVD \\
\midrule
              pvv &          de pvda &      lid ouwehand &              sgp &             sp &          de vvd \\
           de pvv &             pvda &  lid ouwehand nar &           de sgp &          de sp &             vvd \\
      islamitisch &    de partij van &  het lid ouwehand &      sgp fractie &   lid van gerv &  de vvd fractie \\
        lid graus &    van de arbeid &      ouwehand nar &   de sgp fractie &     sp fractie &     vvd fractie \\
    het lid graus &        de arbeid &  ouwehand nar mij &     led dijkgraf &  de sp fractie &       de vvd is \\
    lid graus nar &    partij van de &          ouwehand &  de led dijkgraf &   van gerv nar &          vvd is \\
          miljard &       partij van &       vor de dier &      led van der &       gerv nar &      vor de vvd \\
    graus nar mij &     pvda fractie &           de dier &  de led bisschop &   gerv nar mij &      wat de vvd \\
        graus nar &           arbeid &              dier &     led bisschop &           gerv &     vvd betreft \\
 madlener nar mij &  de pvda fractie &     de partij vor &           sgp is &       van gerv &  de vvd betreft \\
\bottomrule
\end{tabular}
 
\end{table}

\subsection{DV2: Invloed van namen}
In tabel \ref{tab:MostImportantWords} was al te zien dat de meest karakteristieke n-grams voornamelijk achternamen van Kamerleden of partijnamen bevatten. In tabel \ref{tab:rapportwithoutnames} zijn de scores te zien van classificatie met achternamen van Kamerleden en partijnamen vervangen. Deze zijn aanzienlijk lager dan de scores uit deelvraag 1. In tabel \ref{tab:MostImportantWordsWithoutNames} is vervolgens te zien welke n-grams het meest karakteristiek zijn per partij voor deze classificatie.\par
\begin{table}[H]
\caption{Classificatie scores per partij van beste classificatie zonder achternamen van Kamerleden en partijnamen met het relatieve verschil ten opzichte van tabel \ref{tab:classrapport}. Gemiddelde van vijf splitsingen van training en test set.}
\label{tab:rapportwithoutnames}
\centering
\begin{tabular}{lcccc}
\toprule
{} &  Precision &  Recall &  $F_1$ score &  $\Delta F_1$ score (\%) \\
\midrule
SGP          &       0.71 &    0.73 &      0.72 & -18 \\
PvdD         &       0.75 &    0.70 &      0.72 & -19 \\
PVV          &       0.63 &    0.80 &      0.70 & -19 \\
ChristenUnie &       0.68 &    0.46 &      0.55 & -21 \\
CDA          &       0.52 &    0.53 &      0.52 & -23 \\
SP           &       0.54 &    0.71 &      0.61 & -24 \\
D66          &       0.55 &    0.55 &      0.55 & -28 \\
VVD          &       0.54 &    0.49 &      0.52 & -30 \\
50PLUS       &       0.86 &    0.49 &      0.62 & -32 \\
PvdA         &       0.51 &    0.48 &      0.50 & -32 \\
GroenLinks   &       0.64 &    0.38 &      0.48 & -41 \\
\midrule
Totaal       &       0.59 &    0.58 &      0.57 &        -29 \\
\bottomrule
\end{tabular}

\end{table}

\begin{table}[H] 
\caption{Meest karakteristieke n-grams per partij op basis van classificatie uit deelvraag 1 zonder achternamen van Kamerleden en partijnamen gedurende kabinet-Rutte II.} 
\label{tab:MostImportantWordsWithoutNames} 
\centering
\hspace*{-1in}
\begin{tabular}{lllll}
\toprule
                 50PLUS &             CDA &        ChristenUnie &           D66 &              GroenLinks \\
\midrule
                ouderen &  PARTIJ fractie &       vluchtelingen &  mijn fractie &                     zou \\
        gepensioneerden &        inwoners &           inderdaad &          mijn &       kamer hierover te \\
               plussers &          PARTIJ &        mensenhandel &    natuurlijk &        persoonsgebonden \\
 koopkrachtontwikkeling &        regering &              zullen &       fractie &            in elk geval \\
                 oudere &     de regering &            gezinnen &   het kabinet &               elk geval \\
                     50 &            echt &  voedselverspilling &       vandaag &                  in elk \\
            50 plussers &         fractie &          constateer &  buitengewoon &             hierover te \\
              werkenden &            hier &       ik constateer &    belangrijk &     belastingontwijking \\
        overwegende dat &             wij &          onder meer &      minister &             regering om \\
            overwegende &            zeer &         begeleiding &       kabinet &  hierover te informeren \\
\bottomrule
\end{tabular}
 
\end{table} 
\addtocounter{table}{-1} 
\begin{table}[H] 
\caption{Meest relevante n-grams per partij op basis van classificatie uit deelvraag 1 zonder achternamen van Kamerleden en partijnamen gedurende kabinet-Rutte II. \emph{(Vervolg)}} 
\centering
\hspace*{-0.6in}
\begin{tabular}{llllll}
\toprule
              PVV &         PvdA &              PvdD &                    SGP &            SP &            VVD \\
\midrule
      islamitisch &       kinder &              dier &  mevrouw de voorzitter &          zegt &     volgen mij \\
        nederland &       jonger &          de natur &             mevrouw de &       huurder &  yyyyy fractie \\
          miljard &  mijn partij &            de bio &               dank zer &          mens &         volgen \\
          brussel &      gezamen &     bio industrie &                   punt &     de bevolk &          yyyyy \\
 belastingbetaler &        beter &     constater dat &           eenverdiener &        bevolk &         veilig \\
               al &       tevred &               bio &                    wel &        bezuin &       regelgev \\
            islam &      collega &  de bio industrie &            bewindslied &    bestuurder &     essentieel \\
         de islam &          die &            milieu &               allerlei &  bureaucratie &       aangegev \\
          allemal &          sam &         industrie &                     je &          toch &     ondernemer \\
          mijnher &    circulair &         constater &      vor de beantwoord &    mening dat &          kader \\
\bottomrule
\end{tabular}
 
\end{table}

In tabel \ref{tab:rapportonlynames} zijn de scores te zien voor een classificatie met alleen achternamen van Kamerleden en partijnamen. De scores zijn gedaald ten opzichte van de resultaten van deelvraag 1, maar hoger dan die zonder achternamen van Kamerleden en partijnamen.\par
\begin{table}[H]
\caption{Classificatierapport van beste classificatie met alleen achternamen van Kamerleden en partijnamen. Hiervoor is alleen gebruikgemaakt van unigrams. Gemiddelde van vijf splitsingen van training en test set.}
\label{tab:rapportonlynames}
\centering
\begin{tabular}{lrrr}
\toprule
{} &  Precisie &  Sensitiviteit &  $F_1$ score \\
\midrule
50PLUS       &       0.82 &    0.88 &      0.85\\
PvdD         &       0.68 &    0.78 &      0.69 \\
GroenLinks   &       0.71 &    0.66 &      0.68  \\
PVV          &       0.66 &    0.71 &      0.67  \\
CDA          &       0.67 &    0.65 &      0.66  \\
ChristenUnie &       0.66 &    0.58 &      0.62 \\
SP           &       0.61 &    0.64 &      0.62  \\
VVD          &       0.68 &    0.57 &      0.62  \\
SGP          &       0.69 &    0.54 &      0.60  \\
D66          &       0.56 &    0.53 &      0.54  \\
PvdA         &       0.56 &    0.51 &      0.52  \\
\midrule
Totaal       &       0.64 &    0.62 &      0.62  \\
\bottomrule
\end{tabular}

\end{table}

\subsection{DV3: Oppositie of regering}
In figuur \ref{fig:distributies} zijn de distributies van de errors, zoals gedefinieerd in formule \ref{eq:error}, te zijn van combinaties tussen regerings- en oppositiepartijen.
\begin{figure}[H]
    \centering
    \hspace*{-0.2in}
    \subfloat[Tussen twee regeringspartijen]{{\includegraphics[width=7cm]{Verslag/Tables/Regering.png} }}%
    \subfloat[Tussen twee oppositiepartijen]{{\includegraphics[width=7cm]{Verslag/Tables/Oppositie.png} }}\quad
    \hspace*{-0.2in}
    \subfloat[Tussen een regeringspartij en een oppositiepartij]{{\includegraphics[width=7cm]{Verslag/Tables/Mix.png} }}%
    \subfloat[Totaal]{{\includegraphics[width=7cm]{Verslag/Tables/Totaal.png} }}\quad
    \caption{Distributie van de error uit formule \ref{eq:error} voor de verschillende combinaties.}%
    \label{fig:distributies}%
\end{figure}

\begin{table}[H] 
\caption{Meest karakteristieke n-grams per partij op basis van classificatie uit deelvraag 2 gedurende kabinet-Balkenende IV.} 
\label{tab:WoordenBalkenende4} 
\centering
\hspace*{-1in}
\begin{tabular}{lllll}
\toprule
              CDA &        ChristenUnie &              D66 &          GroenLinks &         PVV \\
\midrule
   PARTIJ fractie &  fractie van PARTIJ &          premier &       PARTIJfractie &     burgers \\
              wij &      de fractie van &       de premier &  fractie van PARTIJ &        land \\
          fractie &          de fractie &              hij &             premier &  de burgers \\
       wij hebben &         fractie van &          ik hoop &          de fractie &      burger \\
             dank &       verschillende &            jaren &      de fractie van &        onze \\
         KAMERLID &        mijn fractie &             hoop &         fractie van &        deze \\
          overleg &       beantwoording &           europa &           politieke &      gewoon \\
       aangegeven &                blij &   schone energie &                  ik &  immigratie \\
     buitengewoon &              moment &  de arbeidsmarkt &                 mij &        niet \\
 algemeen overleg &           discussie &          plannen &                deal &  natuurlijk \\
\bottomrule
\end{tabular}
 
\end{table} 
\addtocounter{table}{-1} 
\begin{table}[H] 
\caption{Meest karakteristieke n-grams per partij op basis van classificatie uit deelvraag 2 gedurende kabinet-Balkenende IV.\emph{(Vervolg)}} 
\centering
\hspace*{-0.6in}
\begin{tabular}{lllll}
\toprule
       PvdA &              PvdD &                SGP &            SP &                        VVD \\
\midrule
        wij &            dieren &       mijn fractie &        mensen &                     PARTIJ \\
    vrouwen &     bio industrie &                wel &       leraren &             PARTIJ fractie \\
 belangrijk &  de bio industrie &      beantwoording &          niet &               onze fractie \\
         of &            de bio &  voorzitter ik wil &          zegt &                    fractie \\
  onderzoek &               bio &      de voorzitter &     personeel &                         je \\
      weten &       veehouderij &               toch &    leerlingen &                      markt \\
     vragen &            natuur &   de bewindslieden &            is &  voorzitter PARTIJ fractie \\
        roc &     dierenwelzijn &      bewindslieden &    informatie &                     in elk \\
       goed &    de veehouderij &    dankzeggen voor &     onderwijs &                ondernemers \\
   kinderen &         industrie &         dankzeggen &  bureaucratie &               in elk geval \\
\bottomrule
\end{tabular}
 
\end{table}
In tabel \ref{RegeringOppositie} zijn de resultaten van de classificatiescores te zien waarbij de classificatie getraind is op een zittingsperiode, maar getest op een andere. \par
\begin{table}[H]
\caption{$F_1$ scores van de classificatie getraind op ene zittingsperiode en getest op andere zittingsperiode. Scores van een classificatie getraind en getest op kabinet-Rutte II zonder 50PLUS zijn bijgevoegd ter referentie, als ook de relatieve daling. Classificatiemethode uit deelvraag 1 is gebruikt zonder achternamen van Kamerleden en partijnamen. Partijen met een asterisk zijn gewisseld van partij-status.}
\centering
\label{RegeringOppositie}
\begin{tabular}{llll}
\toprule
{}& {}&\multicolumn{2}{l}{Training set$\rightarrow$ Test set}\\
\midrule
{} &Rutte II &\makecell{Balkenende IV $\rightarrow$ Rutte II\\Baseline = 0.11}&  \makecell{Rutte II $\rightarrow$ Balkenende IV\\Baseline = 0.12}  \\
\midrule
CDA*          &0.53&       0.28 &    0.43  \\
ChristenUnie* &0.55&       0.37 &    0.22  \\
D66          &0.54&       0.16 &    0.28  \\
GroenLinks   &0.49&       0.31 &    0.04  \\
PVV          &0.70&       0.50 &    0.60  \\
PvdA         &0.52&       0.29 &    0.27  \\
PvdD         &0.73&       0.64 &    0.45 \\
SGP          &0.74&       0.56 &    0.49  \\
SP           &0.61&       0.41 &    0.53 \\
VVD*          &0.51&       0.18 &    0.10  \\ \hline
Totaal       &0.58&       0.34 &    0.35 \\
\bottomrule
\end{tabular}

\end{table}


\subsection{DV4: Links of rechts}

\subsection{DV5: Woordgebruik van sprekers}
In tabel \ref{tab:rapporttaalgebruik} staan de scores van classificatie waarbij de Kamerleden verdeeld zijn over de training en test set. De scores zijn hierbij amper hoger dan de baseline.
\begin{table}[H]
\caption{Classificatierapport van beste classificatie met de Kamerleden verdeeld over training en test set. Gemiddelde van vijf splitsingen van training en test set.}
\label{tab:rapporttaalgebruik}
\centering
\begin{tabular}{lrrrr}
\toprule
{} &  Precision &  Recall &  $F_1$ score &  $\Delta F_1$ score (\%) \\
\midrule
50PLUS       &       0.29 &    0.06 &      0.09 &           \\
CDA          &       0.12 &    0.20 &      0.14 &          \\
ChristenUnie &       0.08 &    0.14 &      0.09 &           \\
D66          &       0.22 &    0.22 &      0.22 &          \\
GroenLinks   &       0.16 &    0.04 &      0.05 &          \\
PVV          &       0.29 &    0.50 &      0.37 &          \\
PvdA         &       0.25 &    0.19 &      0.21 &          \\
PvdD         &       0.46 &    0.17 &      0.22 &          \\
SGP          &       0.17 &    0.05 &      0.07 &           \\
SP           &       0.34 &    0.33 &      0.33 &          \\
VVD          &       0.31 &    0.26 &      0.24 &          \\
\midrule
Totaal       &       0.31 &    0.24 &      0.24 &         \\
\bottomrule
\end{tabular}

\end{table}

